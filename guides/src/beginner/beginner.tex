
\makeatletter
%%%%%%%%%%%%%%%%%%%%%%%%%%%%%% Textclass specific LaTeX commands.
\newenvironment{lyxcode}
{\begin{list}{}{
\setlength{\rightmargin}{\leftmargin}
\setlength{\listparindent}{0pt}% needed for AMS classes
\raggedright
\setlength{\itemsep}{0pt}
\setlength{\parsep}{0pt}
\normalfont\ttfamily}%
 \item[]}
{\end{list}}
\newenvironment{lyxlist}[1]
{\begin{list}{}
{\settowidth{\labelwidth}{#1}
 \setlength{\leftmargin}{\labelwidth}
 \addtolength{\leftmargin}{\labelsep}
 \renewcommand{\makelabel}[1]{##1\hfil}}}
{\end{list}}

\makeatother

\newcommand{\gauss}{{\tt gauss}}


\chapter{Getting Started: Setup Guide to \baci{}}


Notes on setting up and running \baci{}, the Finite-Element code
at the Chair of Computational Mechanics. This finite element code had been
formerly known as \ccarat{}, and was renamed to \baci{} in spring 2007. You
can still find its old name at many places.
For information about solving specific types of problems using \baci{}
see the problem introduction guides. For information about the inner
workings of the code and recommendations for developers see the developing
guide.


\section{Getting the code}

The \baci{} code lives in an subversion repository on \gauss{},
the central LNM file server. There are two ways to get the code.

\begin{itemize}
\item Students ask their custodian :) for a copy. Before that can happen
the non-disclosure form needs to be filled.
\item Coworkers get it via subversion:

\begin{lyxcode}
svn~checkout~svn://gauss/trunk/ccarat
\end{lyxcode}
Ask your friendly LNM administrator for a subversion account.

It is a very good idea to get to know subversion before you do this.
There is the book {}``Version Control with Subversion \url{http://svnbook.red-bean.com}''
online.

\end{itemize}

\subsection{svn hints}

Once you have a copy of \baci{} in your home directory and you start
working on it you will need only a small set of subversion commands:

Check the current state of your copy (compared to the version
\baci{} you fetched from the repository) with \texttt{svn~status}, get
new versions with \texttt{svn~update}, mark files and directories to
be added or removed with \texttt{svn~add} and \texttt{svn~remove}
and commit your changes with \texttt{svn~commit}.

Sometimes, however, you need to edit a file that is owned by somebody
else. Somebody you want to was about your changes before you commit
them. To do this you can easily save your changes in a separate file:
\begin{lyxcode}
svn~diff~>~patchfile
\end{lyxcode}
These patches can be applied to a clean version of \baci{} using
\begin{lyxcode}
patch~-p0~<~patchfile
\end{lyxcode}
but can also be viewed conveniently and applied selectively using the
graphical interface \texttt{kompare} from the \texttt{kdesdk} package
\begin{lyxcode}
kompare~patchfile
\end{lyxcode}
All these commands have to be entered in the \ccarat{} directory.

\subsection{Directory structure}

The \ccarat{} code comes with documentation, example input files
and support scripts. The important subdirectories are the following:

\begin{lyxlist}{00.00.0000}
\item [{\texttt{src}}] contains the real \ccarat{} code in several subdirectories
\item [{\texttt{Input}}] contains various valid and running \texttt{{*}.dat}
files, \ccarat{} input files that are used for (automatic) testing.
\item [{\texttt{config}}] contains configuration files (platform specifications)
needed to setup a \ccarat{} Makefile
\item [{\texttt{scripts}}] contains configuration scripts needed to setup
a \ccarat{} Makefile
\item [{\texttt{testing}}] contains scripts used for automatic testing
and the list of files that are to be run during tests
\item [{\texttt{doc}}] contains all the documentation
\item [{\texttt{.}}] there are only a few scripts and two Makefile fragments
in the main directory of \ccarat{}
\end{lyxlist}

\section{Setup and Run}

\ccarat{} is developed and used on Linux. Other Unixes work as well.
Windows versions might be created using cygwin or mingw, but this
will require some small modifications and is not covered here.

\ccarat{} is a non-interactive shell application that reads an input
file and creates a bunch of files in return. To build and run it you
will need a basic understanding of Linux and the Linux shell. You
will also want to choose your favorite text editor. (See section~\ref{sec:Text-editors}.)


\subsection{Configure}

Before compiling a \texttt{Makefile} needs to be generated. There
is a configuration script to do this.

The scripts \texttt{do-configure} and \texttt{do-configure-debug}
generate the \texttt{Makefile} for either the release or the debug
version of BACI. Run the required one in the console.

After reconfiguring use \texttt{make clean} before recompiling

\subsection{Compile}

After the \texttt{Makefile} is there simply run \texttt{make}. You
might want to run \texttt{make all} to create the binary output filters,
too. See below.

The required libraries and include files are found in \texttt{/lnm/lib/\$PLATFORM/\$VERSION}.
\ccarat{} depends on the Trilinos libraries, a few external solver
libraries and metis and hdf5. Additional the boost libraries are used.
You should not need to worry about these libraries
as long as to stick to the provided configurations.

Please note that these are static libraries. \ccarat{} does not need
these libraries to run. The libraries are needed to compile the source
only.


\subsubsection{Details on library setup}

You might want to skip this section.

All libraries needed to compile \ccarat{} on a specific platform
(apart from the system provided ones) are to be in the directory \texttt{/lnm/lib/\$PLATFORM/\$VERSION},
where \texttt{\$PLATFORM} referes to the Linux version used and \texttt{\$VERSION}
referes to the serial or MPI version supported on this Linux.

The currently supported platforms are \texttt{fc5}, \texttt{fc6},
\texttt{debian} (sarge), \texttt{cluster}. The debian version will
be discontinued once all debian workstations are gone. Each platform
supports a serial and one parallel version.

Inside \texttt{/lnm/lib/\$PLATFORM/\$VERSION} there are the directories
\texttt{include}, \texttt{lib}, \texttt{trilinos} and optionally \texttt{trilinos\_dbg}.
There directories contain:

\begin{lyxlist}{00.00.0000}
\item [{\texttt{include}}] include files of the libraries or, if modifications
have been made or the include files are not easy to extract, the whole
source packages
\item [{\texttt{lib}}] compiled libraries. These are aztec 2.1, spooles
2.2, umfpack 4.1, metis 4.0, superlu 2.0, visual2, visual3, lapack,
blas.


Both visual2 and visual3 are optional and are only supported on 32bit
Linux with g77.

\item [{\texttt{trilinos}}] a installed version of the trilinos libraries,
that is the directories \path{/lnm/lib/\$PLATFORM/\$VERSION/trilinos/include}
and \texttt{/lnm/lib/\$PLATFORM/\$VERSION/trilinos/lib} with appropiate
content are required. The packages from trilinos that are needed include
\texttt{amesos}, \texttt{aztecoo}, \texttt{epetra}, \texttt{epetraext},
\texttt{ifpack}, \texttt{loca}, \texttt{ml}, \texttt{moertel}, \texttt{nox},
\texttt{teuchos}.


The trilinos sources reside in \texttt{/lnm/lib/Trilinos}. It is currently
a developer version close to release~7.

\item [{\texttt{trilinos\_dbg}}] a version of the trilinos libraries with
debug symbols enabled.
\end{lyxlist}

\subsubsection{Have a local copy of the \ccarat{} libraries}

Caution: This advanced stuff. You can savely skip it.

The \ccarat{} library organization is meant to allow you to compile
\ccarat{} on your local machine without access to the network directory
\texttt{/lnm}. All you have to do is

\begin{enumerate}
\item copy the appropiate \texttt{/lnm/lib/\$PLATFORM/\$VERSION} directory
to your home directory somewhere and
\item copy and modify the corresponding platform specification file (\texttt{config/\$PLATFORM\_FILE}).
\end{enumerate}
Note: You are urged to make a copy and leave the original platform
specification file as it is. You are not allowed to submit changes
to the platform specification file without consulting the \ccarat{}
maintainer first.


\subsection{Running examples}\label{beginner:sec:running-examples}

In \texttt{Input} there are test examples; all necessary ``packages''
must have been activated in the defines-file that was used to configure
the \ccarat{} at hand. For example,

\texttt{./cca\_fc6\_ser.fast Input/f2\_drivencavity20x20\_drt.dat xxx }

runs the 2d fluid driven cavity example and writes the output to files
beginning with \texttt{xxx}. If the run is interrupted with an error
'\texttt{MAXNOD} is too small' then the values given in a table above
this error message have to be inserted in '\texttt{config/defines.my}';
e.g. \texttt{MAXNOD=20} in second line, \texttt{MAXELE=4} in third
line, \ldots; then recompile and rerun to run a parallel version you
need to use the mpirun command like this:

\begin{lyxcode}
mpirun~-np~1~./cca\_fc6\_par.fast~Input/f2\_drivencavity20x20\_drt.dat~xxx
\end{lyxcode}

\section{Testing}

There is the script \texttt{test\_script} that runs all the examples
file from the \texttt{Input}-directory that are marked for testing.
It knows four modes depending on the command line parameters:

\begin{lyxlist}{00.00.0000}
\item [{\texttt{single~test}}] compile \ccarat{} to fit one input file
exactly and run it
\item [{\texttt{test~all}}] compile \ccarat{} for all input files and
run them one after the other
\item [{\texttt{restart~test}}] compile \ccarat{} for all input files,
run them and restart them from a certain step
\item [{\texttt{release~test}}] compile \ccarat{} so this it is able
to run any (or neary any) input file and run them all using this version
\end{lyxlist}
After each run some result values are compared to expected values.
Only if these values match the test was successful.

The \texttt{test\_release} script can be used to abbreviate all this.
It runs \texttt{test all}, \texttt{restart test} and \texttt{release
test} in a row:

\begin{lyxcode}
./test\_release~config/muench.fc6.ser
\end{lyxcode}
\ccarat{} needs to survive these tests on any supported platform
at any time. This is enforced by nightly tests.

A nightly test builds and runs all configured example files. The BuildBot
server runs on \gauss{}, the clients that do the actual testing are
distributed among the desktop machines.

To add your own input file to the nightly tests edit \texttt{testing/list\_of\_files}.


\section{Cluster}

We own a cluster system for massive parallel processing. The cluster
consists of 54 nodes (two processors each), a frontend and a file
server. For obvious reasons the frontend is called \texttt{hop}, the
file server is called \texttt{malt}.

Your home directory is accessible on all nodes and both server machines.
Additionally there is scratch space in the \texttt{/scratch} directory
on all nodes.

You will normally want to login to the frontend, however if you have
to transfer huge files you are better off accessing the file server
directly.


\subsection{First Time Setup }

Before the cluster can be used for parallel calculations, a few setup
steps have to be performed


\subsubsection{Get an account}

Ask your friendly LNM administrator. Ask politely.


\subsubsection{generate ssh keys for password-less access to all nodes}

\begin{itemize}
\item Login to the cluster with \texttt{ssh username@hop.lnm.mw.tum.de}
\item If you have not done it before, change the default password set by
the Administrator with \texttt{passwd }
\item Create new SSH key with \texttt{ssh-keygen -t rsa} and press enter
without entering a password
\item in the folder \texttt{\textasciitilde{}/.ssh} append the generated key
into the authorized key file with \texttt{cat < id\_rsa.pub >> authorized\_keys}.
This allows to connect to all nodes on the cluster and is essential
for running any kind of job
\end{itemize}

\subsubsection{optinally generate ssh keys for password-less access to the frontend }

To make life pleasant, add the public key also to your \texttt{authorized\_keys}
file on your desktop machine. This way you can login to the cluster
from your account without a password.


\subsection{Getting the code on the cluster}

Use subversion as described above our copy with \texttt{scp}, whatever
you like best.


\subsection{Running a parallel job}

There is a queue where you submit your job, that is the executable
and its input file. To do this you will have to write a short job
file that tells the queue where to find and how to run your code.

The following file prototype serves as the execution script for the
cluster. Copy and modify it to your needs.

\begin{lyxcode}
\#!/bin/sh

\#~Here~are~some~comments~the~PBS~system~will~search~for.

\#...job~name~

\#PBS~-N~jobname~

\#...nodes~=~number~of~nodes,~ppn~=~processors~per~node~(always~2)

\#PBS~-l~nodes=4:ppn=2



\#~we~start~our~job~from~inside~the~source

\#~directory~and~we~also~want~the~results~there.

srcdir=\$PBS\_O\_WORKDIR~

cd~\$srcdir



\#~number~of~processors

PROCS=`wc~-l~<~\$PBS\_NODEFILE`~



\#~parameters,~to~be~changed!

PREFIX=xxx~

EXE=cca\_par\_ompi\_cluster.fast

INPUT=inputfile.dat

RESTART=



\#~add~the~InfiniBand~library~to~the~search~path

VAPILIB=/usr/local/ibgd/driver/infinihost/lib64~

export~LD\_LIBRARY\_PATH=\$LD\_LIBRARY\_PATH:\$VAPILIB



MPIDIR=/cluster/openmpi-1.1.4



\$MPIDIR/bin/mpirun~-np~\$PROCS~-hostfile~\$PBS\_NODEFILE~\$EXE~\textbackslash

~~~~\$INPUT~\$PREFIX~\$RESTART~|~tee~\$srcdir/\$PREFIX.log


\end{lyxcode}
This job file makes several assumptions:

\begin{itemize}
\item You want to run a parallel job compiled using OpenMPI (the supported
MPI version on the cluster.)
\item You are going to use 4 nodes with 2 processors each. If that is not
what you want change the line

\begin{lyxcode}
\#PBS~-l~nodes=4:ppn=2
\end{lyxcode}
Note: You will always end up with two processes on each node unless
you provide your own nodefile. Most of the time two processes per
node are exactly what you want.

\item You want the output files to go to the same directory where your \ccarat{}
executable resides. If you build \ccarat{} with the \texttt{BINIO}
flag set and switched binary output on in your input file, this directory
must be accessible from all nodes. Most probably you will want to
start the job from somewhere inside your home directory.


Note that you can use the scratch space for output. This requires
a slight change in the job file above. Additionally the binary output
does not yet support writing processor local files.

\item The \texttt{PREFIX} and \texttt{INPUT} variables are most likely to
need a change.
\item If you want to restart a calculation you have to set the \texttt{PREFIX}
variable accordingly and specify \\
\texttt{RESTART=restart}. Additionally
the restart step must be set inside the input file.
\end{itemize}
To submit the job use \texttt{qsub job.sh}, with the above file saved
as \texttt{job.sh}. Each job in the queue gets an unique number.


\subsubsection{Useful queue commands}

\begin{lyxlist}{00.00.0000}
\item [{\texttt{qsub~job.sh}}] submit a job
\item [{\texttt{qsub~-I}}] start an interactive session on a node
\item [{\texttt{qstat}}] show currently running (and waiting) jobs
\item [{\texttt{qstat~-f~{[}number]}}] show details to all (or one) job
\item [{\texttt{showq}}] another way to view the queue
\item [{\texttt{qdel~number}}] remove a currently running job
\end{lyxlist}

\subsubsection{Hints}

\begin{itemize}
\item You are allowed to login to any node that is used by your job via
\texttt{ssh}.
\item You can get a nice view on the current cluster performance by browsing
to ganglia\\
\url{http://hop.lnm.mw.tum.de/ganglia/}.
\item The \texttt{\#PBS} comments in the job file can be given as command
line arguments to \texttt{qsub} as well. Without the \texttt{\#PBS}
string.
\end{itemize}

\section{Preprocessing}

There are not so many means to create a valid input file.


\subsection{CUBIT}\label{beginner:sec:cubit}
CUBIT (http://cubit.sandia.gov/) is a powerfull  pre- postprocessing tool. CUBIT can create EXODUS-II files
which can be converted into a valid BACI inpufile using the pre\_exodus filter, see section \ref{beginner:sec:preexodus}.

\subsubsection{How to start CUBIT}
CUBIT is installed on Gauss. Its folder is \texttt{/lnm/programs/cubit12/}
You can start CUBIT by typing  \texttt{/lnm/programs/cubit12/cubit} on any LNM machine.
(It may be convenient to place a link to this executable in your \texttt{\~{/}bin}
directory by doing \texttt{ln -s /lnm/cubit12/cubit \~{/}bin/cubit}, then
typing \texttt{gid} will do the same. Another optin is to create an alias in your \texttt{\~{/}.bashrc} file)
Once started go to \textit{Help} $\rightarrow$ \textit{Cubit Tutorials} for an introduction or go to Chapter
\ref{tut_fluid_preexo:chap}.


\subsection{Modify \ccarat{} input files}

\ccarat{} input files are text files so you can modify them using
your favorite text editor. However, sometimes you might want some
more modifications (e.g. moving many nodes coordinates) that might
be better done by a script. And indeed there is a python script that
can help you edit input files. (ToDo: There is no central place yet.)


\subsection{viewer}

There is a small but useful \texttt{python} script that allows to
visualize \ccarat{} input files (meshes and conditions). It is in
the \texttt{viewer} directory inside the repository. To run it you
will need to have \texttt{PyQt} and \texttt{PyOpenGl} installed. You
can get it with

\begin{lyxcode}
svn~co~svn://gauss/trunk/viewer
\end{lyxcode}

\subsection{pre\_exodus}\label{beginner:sec:preexodus}
This filter creates (and validates) a \baci{} input file from an EXODUS-II file
generated by one of the meshing tools CUBIT, ICEM-CFD or Harpoon. Type
\texttt{make pre\_exodus} in your \baci{} folder to compile it.
Have a closer look at the chapter \emph{FSI Tutorial 3d with pre\_exodus and Cubit}
to learn more about its usage.
This new tool replaces the former \emph{exofilter} which is still available at
\begin{lyxcode}
svn~co~svn://gauss/trunk/exofilter
\end{lyxcode}
Note: Paraview can directly visualize EXODUS-II files and can be used
as a viewer to check your mesh and defined node sets before using the
preprocessing tool.


\section{Postprocessing}

\baci{} supports traditional text file output and more recent binary
output. The later needs to be postprocessed by some filter before
it can be viewed.


\subsection{ParaView}

By applying the filter \texttt{post\_drt\_ensight}
to your binary \baci{} result data, a \emph{*.case} file is created.
This result format can then be loaded into \emph{paraview} for visualization.
Start paraview with the command\\
\texttt{/lnm/programs/paraview-3.0.2-Linux-x86/bin/paraview}
A ParaView tutorial is available in the internet:
\begin{verbatim}
http://www.paraview.org/Wiki/The_ParaView_Tutorial
\end{verbatim}

\subsection{Ensight}

By applying the filter \texttt{post\_drt\_ensight} (see next section)
to your binary \baci{} result data, a \emph{*.case} file is created.
This result format can then be loaded into \emph{Ensight} for visualization.
Start Ensight with the command
\texttt{/lnm/programs/CEI/bin/ensight8}\\


\subsection{Filters}

All filters read the control files \ccarat{} creates. Common filter
options are:

\begin{lyxcode}
./post\_filter~{[}-s~start:end{[}:step]]~control-file
\end{lyxcode}
Process output steps from \texttt{start} to \texttt{end} every \texttt{step}.
Works on real time steps, steps not written by ccarat are counted,
too. Both \texttt{start} and \texttt{end} can be empty, in which case
the filter will process from the first and to the last step, respectively.

Common code for output filters is in \texttt{src/post\_common}.

Note: filters are largely independent of ccarat and can be enhanced
on demand.


\subsubsection{post\_out}

Generate readable output. Unprocecced numbers. No known issues.


\subsubsection{post\_visual2}

interactive visual2 viewer. 32bit g77 linux only. 2d only. Fluid and
FSI supported.

Issues: Library buggy and unsupported, console input not working.


\subsubsection{post\_visual3}

interactive visual3 viewer. 32bit g77 linux only. 2d and 3d. Fluid
and FSI supported.


\subsubsection{post\_monitor}

generate gnuplot files for values at selected dofs:

\begin{lyxcode}
./post\_monitor~{[}options]~control-file~monitor-file
\end{lyxcode}
with a monitor-file in control-file format consisting of at least
one block:

\begin{lyxcode}
monitor:

~~~field~=~\char`\"{}fluid\char`\"{}

~~~field\_pos~=~0

~~~discretization~=~0

~~~node~=~440

~~~group~=~\char`\"{}velocity\char`\"{}

~~~dof~=~0

~~~dof~=~1
\end{lyxcode}
Here the three variables \texttt{field}, \texttt{field\_pos} and \texttt{discretization}
specify the discretization that is to be used. \texttt{node} gives
the global node id. \texttt{group} gives the result name that should
be monitored. And any number of \texttt{dof} variables (in the example
two dofs are set, no mistake!) gives the dofs at the node those values
are to be plotted.


\subsubsection{post\_file\_manager}

copy, merge and slice binary output files:

\begin{lyxcode}
./post\_file\_manager~control-file~new-control-file
\end{lyxcode}
Utility filter by Michael Geppert.


\subsubsection{post\_ensight}

generate ensight (paraview) input files:

\begin{lyxcode}
./post\_ensight~control-file
\end{lyxcode}
supported problem types: 2d and 3d fluid and fsi another Michael Geppert
filter.


\subsubsection{post\_cmp}

compare two binary fluid results, assuming the same mesh was used.
Create difference of velocity:

\begin{lyxcode}
./post\_cmp~control-file1~control-file2
\end{lyxcode}
Issues: very simple tool. Just for velocity of fluid problems. Enhancements
are possible.

\subsubsection{post\_drt\_ensight}
generate ensight (paraview) input files of \baci{} data: For
instance, if you
called your output file base name \texttt{xxx}, you execute
\begin{lyxcode}
./post\_drt\_ensight --file="xxx"
\end{lyxcode}
Additional filter options can be made visible by
\begin{lyxcode}
./post\_drt\_ensight --help
\end{lyxcode}
With the help of a small tool called \texttt{ens\_checker} (available via
 \texttt{/lnm/programs/CEI/bin/ens\_checker})\\
it is also possible to check whether the generated filter output fulfills the ensight format.\\
\\
If you configure \baci{} for parallel runs, also the post filters
are configured with parallel layout.
The parallel filter version is especially recommended for computations based on huge discretizations
(more than 1 mio. elements as a rough rule of thumb). The main idea is to have the discretization distributed over several processors and hence a distributed consumption of RAM preventing a possible swapping on a single machine.
On your local machine you can execute the parallel filter on \emph{numprocs} processors with
\begin{lyxcode}
lamboot\\
mpirun~-np~\emph{numprocs}~post\_drt\_ensight --file="xxx"
\end{lyxcode}
Note, that also the call
\texttt{./post\_drt\_ensight\_par --file="xxx"}
is still valid.\\
After computing "big jobs" on the cluster it is \emph{strongly} recommended that you keep the \baci{} output where it is (on the cluster!)
and filter your data there, too. Finally, the filtered dataset can be exported for visualization.\\
An example job file for executing the parallel ensight filter on \emph{hop} is given below:
\begin{verbatim}
 #!/bin/sh

#PBS -N ensightfilter_par
#PBS -l nodes=3:ppn=2
#mkdir -p /scratch/$USER/benchmark
#cd /scratch/$USER/test

srcdir=$PBS_O_WORKDIR
cd $srcdir

PROCS=6
PREFIX=xxx
FILE=--file=$PREFIX
EXE=post_drt_ensight_par
START=--start=0
END=--end=10
STEP=--step=1
STEPPING=$START" "$END" "$STEP
OUTNAME=yyy
OUTPUT=--output=$OUTNAME
echo $PBS_NODEFILE
VAPILIB=/usr/local/ibgd/driver/infinihost/lib64
export LD_LIBRARY_PATH=$LD_LIBRARY_PATH:$VAPILIB

# use newest version of openmpi
MPIDIR=/cluster/openmpi-1.1.4
# use gcc-4.2.2 compiler
CPPLIB=/cluster/gcc-4.2.2/lib64
export LD_LIBRARY_PATH=$LD_LIBRARY_PATH:$CPPLIB

# execute the filter
$MPIDIR/bin/mpirun -np $PROCS -hostfile $PBS_NODEFILE $srcdir/$EXE $FILE $STEPPING \
 $OUTPUT | tee $PREFIX.filter.log
\end{verbatim}


\section{Development environment}

The main development tools for \ccarat{} are \texttt{gcc}, \texttt{g77},
\texttt{gdb}, \texttt{bash} and the related tools. Most people, however,
prefere some kind of environment that plugs these tools together.

This section does only provide some hints what is there. These tools
are not necessarilly installed on your machine. But they can be using
\texttt{yum} or \texttt{apt-get}.


\subsection{Text editors \label{sec:Text-editors}}

That is a matter of taste, really. Only those with taste choose the
right one: \texttt{xemacs}. However, those of us with taste and style
choose \texttt{emacs}.


\subsubsection{xemacs}

The right one.

A text editor that has shortcuts for everything ever imagined. Very
huge. Comes with extensive documentation. Contains the elisp programming
language for customization and extention. Has been around for ages.
Is a standard on its own. It does not stick to any standard you are
likely to know. Love it or hate it.

It takes some effort to become fluent with xemacs. You have to remember
some seemingly random keyboard shortcuts. The benefits are unimagined
text editing power. Some clever customizations are available that
make editing with xemacs even more fun. (Derived from the xemacs customizations
used by the KDE people.)

A very good reason why you should give xemacs a closer look is the
\texttt{xref} utility.


\paragraph{xref}

is a cross referencing and refactoring tool to be used in connection
with (x)emacs. With this tool you can jump to the definition of any
symbol in your code. It is amazing...

\texttt{xref} is located in \texttt{/lnm/programs/xref}. It needs
to be installed before it can be used. To do so call the \texttt{xrefsetup}
script. It will modify your xemacs configuration file \texttt{\textasciitilde{}/.xemacs/init.el}
and offer a simple tutorial. Do it.

The \texttt{xref} configuration lives in the \texttt{\textasciitilde{}/.xrefrc}
file. For all projects you want to work on (e.g. all copies of \ccarat{}
you happen to keep) you will need a section in that file. A \texttt{\textasciitilde{}/.xrefrc}
file with just one section that covers the \ccarat{} copy located
in \texttt{\textasciitilde{}/ccarat} looks like this:

\begin{lyxcode}
-license=0/0/0/5:Lehrstuhl-f\"ur-Numerische-Mechanik:b8f7c7b28aaf7b11



{[}/home/username/ccarat]

~~//~~input~files~and~directories~(processed~recursively)

~~/home/username/ccarat/src/

~~//~~directory~where~tag~files~are~stored

~~-refs~/home/username/Xrefs/ccarat

~~//~~split~tag~files~using~first~letter

~~-refalphahash

~~//~~include~directories

~~-I~/lnm/lib/fc6/ser/include

~~//~resolve~symbols~using~definition~place

~~-exactpositionresolve

~~//~~setting~for~Emacs~compile~and~run

~~-set~compilefile~\char`\"{}mpicc~-g~-ansi~-Wall~\%s\char`\"{}

~~-set~compiledir~\char`\"{}mpicc~-g~-ansi~-Wall~{*}.c\char`\"{}

~~-set~compileproject~\char`\"{}

~~~~~~~cd~/home/kuettler/fem/ccarat

~~~~~~~make

~~~~~~~\char`\"{}

~~-set~run1~\char`\"{}cca\_par\_linux.debg\char`\"{}

~~-set~run2~\char`\"{}cca\_seq\_linux.debg\char`\"{}

~~-set~run5~\char`\"{}\char`\"{}~//~an~empty~run;~C-F8~will~only~compile

~~//~~set~default~to~run1~~~-set~run~\$\{run1\}

~~//~~HTML~configuration

~~-htmlroot=/home/kuettler/HTML

~~-htmlgxlist~-htmllxlist~-htmldirectx~-htmllinenums

~~-htmltab=8~-htmllinenumcolor=000000

~~//~~pre-processor~macros~and~passes~specification

~~-DAZTEC\_PACKAGE

~~-DD\_MLSTRUCT

~~-DBINIO

~~-DDEBUG

~~-DD\_ALE

~~-DD\_AXISHELL

~~-DD\_BEAM3

~~-DD\_BRICK1

~~-DD\_CONTACT

~~-DD\_FLUID

~~-DD\_FLUID\_PM

~~-DD\_FLUID2

~~-DFLUID2\_ML

~~-DD\_FLUID2TU

~~-DFLUID3\_ML

~~-DD\_FLUID2\_PRO

~~-DD\_FLUID3

~~-DD\_FLUID3\_F

~~-DD\_FSI

~~-DD\_INTERF

~~-DD\_MAT

~~-DD\_OPTIM

~~-DD\_SHELL8

~~-DD\_SHELL9

~~-DD\_WALL1

~~-DD\_WALLGE

~~-DLINUX\_MUENCH

~~-DPERF

~~-DRESULTTEST

~~-DS8CONTACT

~~-DUMFPACK

~~-DSOLVE\_DIRICH

~~-DSOLVE\_DIRICH2

~~-pass1

~~~~-DPARALLEL

~~~~-DSPOOLES\_PACKAGE

~~-pass2

~~~~-DNOTPARALLEL~
\end{lyxcode}
Not all of that is important. In particular all the settings related
to compile, run and html are of no importance here. You might try
to find an use for that if you like. Three parts, however, are highly
relevant:

\begin{itemize}
\item the license string in the first line of the file
\item the directories that state where your project lives, where the project
sources are, where to put the internal database and where to find
additional include files
\item the list of preprocessor macros that are defined {}``on the command
line.''
\end{itemize}
After you setup a project like shown above start your xemacs and choose
{}``Create Xref Tags'' from the {}``Xrefactory'' menu. This is
going to take a while. After that you can jump around in \ccarat{}
just like you did in the small \texttt{xref} tutorial. Cool! And it
helps a lot!

\subsubsection{post\_drt\_monitor}

This filter \texttt{post\_drt\_monitor} is used to extract the behaviour of a
single node by writing a
text file \texttt{xxx.mon} which contains a table like
for example
\begin{verbatim}
# structure problem, writing nodal data of node 1
# control information: nodal coordinates   x = 1    y = 0    z = 0
#
# step   time     d_x      d_y      v_x      v_y
   1    0.05  0.00615565   0.246226   9.84904
   2    0.1  0.0244717   0.486418   -0.24137
   3    0.15  0.0544966   0.714575   9.36765
\end{verbatim}
This file can be readily digested by many plotting applications:
\emph{GNUplot} is perfectly suitable.

The filter is applied by
executing for instance
\begin{verbatim}
./post_drt_monitor --field="structure" --node=1 --file="xxx"
\end{verbatim}

You can get all options by asking for \texttt{--help}, i.e.\@
\begin{verbatim}
./post_drt_monitor --help
\end{verbatim}


The filter is build by running a
\begin{verbatim}
make post_drt_monitor
\end{verbatim}


\subsubsection{emacs}

A close miss. Everything said in faviour of xemacs applies here as
well, except ... it misses an \texttt{x}.


\subsubsection{kdevelop}

The developement environment made by the KDE people. Is meant to be
a modern IDE. There are still problems with the way we use define
flags in \ccarat{}. The buildin parser fails to see many files, so
function based navigation is rather limited.


\subsubsection{nedit, kwrite, kate}

Plain editors. Most often syntax highlighting is supported, so at
least you get colour. All the developement has to be done using the
Linux shell.


\subsection{File comparators}

A file comparator allows to show neatly the differences of two files. These
are handy to compare two (or more) text files such as input files, output
files or source code.

\subsubsection{kompare}

The standard KDE tool for the job.

\subsubsection{Eclipse}

built in

\subsubsection{Smartsvn}

Smartsvn (\texttt{smartsvn}) contains a file comparator against personal
modified and committed versions. Ans more...

\subsubsection{Kdiff3}

kdiff3 -- compares two or three input files or directories

\subsubsection{Xxdiff}

Sometimes it is helpful --indeed very helpful-- to compare two text
files and see their differences. One way to get these differences
displayed conveniently is Xxdiff (\texttt{xxdiff} on the command line).

Xxdiff is a free clone of the Unix freeware Xdiff (http://reality.sgiweb.org/rudy/xdiff/);
due to its heritage it looks a bit siliconish.






\subsection{Debugging frontends}

There is only one debugger: \texttt{gdb}. But it has a very plain
user interface. Most people use a frontend application that internally
talks to \texttt{gdb}.

Development environments that are more than text editors (xemacs,
eclipse, kdevelop) contain a debugging ability. There are, however,
standalone debugging tools. The point of these frontends is to show
the code along with the current position, the contents of variables
and so on.


\subsubsection{ddd}

Debugger frontend with some graphical ability. Based on motiv, therefore
it is sometimes cumbersome to use.


\subsubsection{kdbg}

A KDE frontend to gdb. Lacks the ability to input commands at the
\texttt{gdb} command line. Suitable for simple debugging jobs.


\subsubsection{gdbtui}

Simple \texttt{gdb} enhancement. Text based. Nothing more that \texttt{gdb}
with a source code view.


\subsubsection{ddt}

To debug in parallel on the cluster use \texttt{ddt}. It is installed
in \texttt{/cluster/allinea/ddt} and supports up to 8 processes. There
is one debugging session allowed at a time.

Before you can use \texttt{ddt} you have to set it up so that it knows
you are running your job via OpenMPI and a PBS queue. Several steps
are required:

\begin{itemize}
\item add the line

\begin{lyxcode}
export~DDTMPIRUN=/cluster/openmpi-1.1.4/bin/mpirun
\end{lyxcode}
to your \texttt{\textasciitilde{}/.bashrc}. This allows ddt to run
proccesses in parallel on the frontend. (You are not supposed to do
that regularly.)

\item create a file \texttt{\textasciitilde{}/.ddt/config.ddt} with the
following content

\begin{lyxcode}
DDT~Config~File

{[}editor]~

tab~size~=~4

{[}fonts]~

family~=~fixed~

size~=~14

{[}groups]~

filename~=

{[}startup]~

default~number~of~processes~=~2~

default~debugger~=~gdb~

stop~at~exit~=~no~

stop~at~abort~=~yes

{[}mpi]~

type~=~OpenMPI

{[}queue]~

use~queue~=~yes~

template~file~=~/home/username/pbs.qtf~

submit~=~/cluster/torque/bin/qsub~

cancel~=~/cluster/torque/bin/qdel~JOB\_ID\_TAG~

job~regexp~=~{[}0-9]+~

display~=~/cluster/torque/bin/qstat~

use~num\_nodes~=~yes~

procs\_per\_node~=~2

{[}arguments]~

mpi0~=

{[}attach]~

node~list~file~=~

hide~forked~children~=~yes

{[}attach~list]

{[}memory~debugging]~

enabled~=~yes~

setting~=~none~

interval~=~-1~

preload~=~dmalloc~
\end{lyxcode}
This file can be generated using the \texttt{ddt} gui as well, however
it is far more convenient to create beforehand. This configuration
file tells \texttt{ddt} how to use the queue. The only line you have
to change in the above file is the one defining the \texttt{template
file}.

\item The \texttt{template file} is used to create a PBS job file. There
are some tags that \texttt{ddt} replaces with its values before it
submits the file into the queue. A valid template file looks like
this:

\begin{lyxcode}
\#!/bin/bash

\#~DDT~will~generate~a~submission~script~from~this~by~replacing~these~tags:~

\#~~~~~~~~TAG~NAME~~~~~~~|~~~~~~DESCRIPTION~~~~~~~~~~~~~~|~~~~~~EXAMPLE~

\#~-{}-{}-{}-{}-{}-{}-{}-{}-{}-{}-{}-{}-{}-{}-{}-{}-{}-{}-{}-{}-{}-{}-{}-{}-{}-{}-{}-{}-{}-{}-{}-{}-{}-{}-{}-{}-{}-{}-{}-{}-{}-{}-{}-{}-{}-{}-{}-{}-{}-{}-{}-{}-{}-{}-{}-{}-{}-{}-{}-{}-{}-{}-{}-{}-{}-{}-{}-{}-{}-{}-{}-{}-{}-{}-~

\#~PROGRAM\_TAG~~~~~~~~~~~|~target~path~and~filename~~~~~~|~/users/ned/a.out~

\#~PROGRAM\_ARGUMENTS\_TAG~|~arguments~to~target~program~~~|~-myarg~myval~

\#~NUM\_PROCS\_TAG~~~~~~~~~|~total~number~of~processes~~~~~|~16~

\#~NUM\_NODES\_TAG~~~~~~~~~|~number~of~compute~nodes~~~~~~~|~8~

\#~PROCS\_PER\_NODE\_TAG~~~~|~processes~per~node~~~~~~~~~~~~|~2~

\#~

\#~Note~that~NUM\_NODES\_TAG~and~PROCS\_PER\_NODE\_TAG~are~only~valid~if~DDT~is~

\#~set~to~'use~NUM\_NODES'~in~the~queue~options.~If~not,~they~will~be~replaced~

\#~with~the~number~of~processes~and~1~respectively.

\#PBS~-l~walltime=24:00:00,nodes=NUM\_NODES\_TAG:ppn=PROCS\_PER\_NODE\_TAG~

\#P~BS~-q~debug~

\#PBS~-V~

\#PBS~-o~PROGRAM\_TAG-ddt.output~

\#PBS~-e~PROGRAM\_TAG-ddt.error



srcdir=\$PBS\_O\_WORKDIR~cd~\$srcdir

VAPILIB=/usr/local/ibgd/driver/infinihost/lib64

export~LD\_LIBRARY\_PATH=\$LD\_LIBRARY\_PATH:\$VAPILIB

MPIDIR=/cluster/openmpi-1.1.4

\$MPIDIR/bin/mpirun~-hostfile~\$PBS\_NODEFILE~-np~NUM\_PROCS\_TAG~\textbackslash

~~~~/cluster/allinea/ddt/bin/ddt-debugger~PROGRAM\_ARGUMENTS\_TAG~|~tee~debug\_output
\end{lyxcode}
\end{itemize}
Once you have the configuration in place you can call \texttt{ddt}
via

\begin{lyxcode}
/cluster/allinea/ddt/bin/ddt~cca\_par\_ompi\_cluster.debg~inputfile.dat~xxx
\end{lyxcode}
Note: \texttt{ddt} has a memory debugging ability that consumes a
large amount of memory. If you want to debug large examples you will
have to turn memory debugging off.


\subsection{Debugging hints}

What to do when \ccarat{} fails.


\subsubsection{debug version of \ccarat{}}

To be able to debug \ccarat{} you have to build it in debug mode.
To do this you have to uncomment the \texttt{DEBUG} flag in your \texttt{config/defines.my}
file and call \texttt{configure} again. Do not forget to \texttt{make
clean} before you compile again!

Note: The \texttt{DEBUG} flag slows things down. You are not supposed
to run long jobs with the debug version of \ccarat{}.

Note: If all you want is add the \texttt{DEBUG} flag to your configuration,
there is no need to actually edit your \texttt{config/defines.my}
file. Instead you can set the environment variable \texttt{DEBUG}
to \texttt{yes} and reconfigure:

\begin{lyxcode}
DEBUG=yes~make~reconfig~\&\&~make~clean~\&\&~make
\end{lyxcode}
\texttt{make reconfig} is an abbreviation for the \texttt{configure}
call you used to configure in the first place. In particular both
configuration files will be read again, so if these files have changed
it will do something different.


\subsubsection{build in a different directory}

A \ccarat{} reconfigure overwrites the \texttt{Makefile}. This might
not be what you want. Instead you might run \texttt{configure} from
a directory different from the \ccarat{} source directory. This way
you will get a Makefile in the directory you run configure from. For
example, starting from the \ccarat{} source directory you might:

\begin{lyxcode}
mkdir~quicktest

cd~quicktest

../configure~../config/muench.fc6.ser~../config/defines.my

make
\end{lyxcode}
This will create a \texttt{Makefile} and from that a whole new \ccarat{}
inside the \texttt{quicktest} directory. The \texttt{Makefile} in
the source directory as well all the object files from your original
build will remain untouched. All new files will be inside the \texttt{quicktest}
directory. So you can remove the whole thing by

\begin{lyxcode}
cd~..

rm~-r~quicktest
\end{lyxcode}
Note: Sometimes you want to create an executable fast for just one
run, without worring about source dependencies. So you can skip the
dependency generation by setting the environment variable \texttt{NODEPS}
to \texttt{yes} like this:

\begin{lyxcode}
NODEPS=yes~../configure~../config/muench.fc6.ser~../config/defines.my
\end{lyxcode}

\subsubsection{core files}

To get information out of crashing \ccarat{} processes you want to
switch on \texttt{core} files. To do this add the line

\begin{lyxcode}
ulimit~-c~unlimited
\end{lyxcode}
to your \texttt{\textasciitilde{}/.bashrc} and start a new shell.
This will create a file \texttt{core} or \texttt{core.\$\$} in the
current directory each time \ccarat{} crashes, where \texttt{\$\$}
stands for the process id. Afterwards you can have a look into \ccarat{}
at just the moment the program failed:

\begin{lyxcode}
gdb~cca\_fc6\_ser.debg~core
\end{lyxcode}
Debugger frontends can be used as well.

You might also want to include the \texttt{DSERROR\_DUMP} flag in
your \texttt{config/defines.my} file to have \ccarat{} crash each
time a runtime error occurs (and \texttt{dserror} gets called.)

Note: Writing a \texttt{core} file in Linux is pretty fast. However,
if your process took a large amount of memory and you run it on a
nfs mounted directory, it might take a few seconds to dump the \texttt{core}
file.


\subsubsection{valgrind}

If it gets all confused and you have lost track of your pointers it
might be time to reach for something strong. Enter \texttt{valgrind}.

\texttt{valgrind} is a memory checker (and more!) It simulates your
code, keeps track of the memory usage and reports anything strange
like using unallocated memory, uninitialized variables, freeing the
same pointer twice or leaking memory. \texttt{valgrind} can save lifes!

To use it simple write valgrind before the \ccarat{} executable:

\begin{lyxcode}
valgrind~./cca\_fc6\_ser.debg~Input/f2\_drivencavity20x20.dat~xxx
\end{lyxcode}
The messages from \texttt{valgrind} might seem cryptic. But do not
ignore them. Decipher!

There are two issues with \texttt{valgrind}: It is slow. And memory
demanding. So you will want to use it on small examples.


\subsection{Graphical Developement Environment Eclipse \& CDT}

...includes most of the seperate tools from above.

\subsubsection{Install}

The newest version including preinstalled CDT (C/C++ Development Environment plugin) and Subclipse (Subversion Plugin) is installed at \verb|/lnm/programs/eclipse_cdt|.

\subsubsection{Setup/First Run}
Start Eclipse from the command line by \verb|/lnm/programs/eclipse_cdt/eclipse| (you can make a button for KDE yourself).
You will be prompted to enter the workspace directory. Leave it as recommended and press OK.

\subsubsection{Checkout BACI}


\begin{itemize}
\item \texttt{File -> New -> Project...}
\item  \texttt{SVN -> Checkout Projects from SVN}
\item  \texttt{Create a new repository location}
\item  Url: \texttt{svn://username@gauss.lnm.mw.tum.de/trunk/}

\item Select ccarat (to be renamed to BACI)

\item Click Finish

\item Select Wizard: \texttt{C++ -> C++ Project}

\item Give Projectname: baci

\item Click Finish and then OK

\item Change to console and configure BACI as usual

\item Now go back into eclipse

\item If you are not already there, choose "Workbench".

\item Right-Click on BACI in the Project Explorer and Choose \texttt{Properties}

\item On \texttt{C/C++ General} in "Paths and Symbols":
	In the "Includes" as well as in the "Symbols" part click "Add":
	Write some dummy name in the new window and click at "Add to all configurations" and "Add to all languages"

\item On \texttt{C/C++ Build} De-Select "Generate Makefiles automatically"

\item From the text field "Build directory", remove the part \texttt{/Debug}

\item Open tab "Behaviour" and remove "all" from "Build(Incremental build)" and press "OK"

\item Close Eclipse

\item Change to console and reconfigure BACI (the "DEFINES" are loaded now into BACI)

\item build BACI using "make"

\item Restart eclipse!

\item Happy development...

\end{itemize}

\subsubsection{Usage}

Refer to the experienced and friendly Eclipse community at LNM or the help in Eclipse and the web.

\newpage

\subsection{Checkout student version of BACI (version controlled: git)}
Incorporating the work of students into the current version of BACI or updating the students version of the code can be painful. Using git version control alleviates the pain
to some extend and offers a more convenient way to provide students with BACI updates or to merge their work with the current revision.
However, because svn and git do not really go great together, it takes some effort to set it up. First you will have to create an additional instances of BACI on your Machine. It is as usual
checked out using svn into e.g /Baci\_stud. Afterwards you can initialize a new git repository in /Baci\_stud. From this git instance
of BACI you can now push a version into the student's home directory which is from now on referred to as /Baci\_bare. The students can now pull a version of Baci\_bare onto their local 
drive i.e /scratch to work with/compile BACI. Also they can commit changes to their /Baci\_bare instance, from which you can pull it back up onto your machine.
Allright, here's what you need to do:\\
%\subsubsection{git}

{\bf {If you find any errors or inaccuracies, please inform me (A. Ehrl) or update it on your own!}}

\paragraph{In your home directory}

\begin{enumerate}
  \item check out a new svn version of BACI (in a new folder: e.g. Baci\_stud)
  \item initialize a new git repository in the folder Baci\_stud/  \\
  \texttt{git init}
  \item add all files in the folder Baci\_stud to the repository: \\
  \texttt{git add .}
  \item commit the changes: \\
  \texttt{git commit -a}
  \item generate a new branch \\
  \texttt{git branch <student\_name>}
  \item change to new branch: \\
 \texttt{git checkout <student\_name>}
\end{enumerate}

\paragraph{In students home directory}

\begin{enumerate}
  \item clone the repository Baci\_stud: \\
  \texttt{git clone -$\,$-bare -b <student\_name> $\backslash$ \\
    \phantom{aaaa} ssh://<myname>@<mycomputer>.lnm.mw.tum.de/$\sim$/<myDirectory>/Baci\_stud Baci\_bare}
  \item delete master branch (if there is one): \\
  \texttt{git branch $\rightarrow $ } list all branches in the current clone \\
  \texttt{git branch -d master} \\
\end{enumerate}

\paragraph{In the student's scratch}

\begin{enumerate}
  \item generate some folder structure:
\begin{itemize}
  \item \texttt{mkdir -p /scratch/<student\_name>/workspace}
  \item \texttt{cd /scratch/<student\_name>}
  \item Note: do not change into the workspace
\end{itemize}
  \item clone baci\_bare to this folder (it will be your future workspace): \\
  \texttt{git clone -b <student\_name> $\sim$/Baci\_bare baci}
  \item start eclipse, select the proper workspace (/scratch/<student\_name>/workspace) and create and empty C++ project
  \item Import /scratch/<student\_name>/baci into the new project, which will create a copy in the workspace
  \item delete the original baci:\\
  \texttt{rm -rf /scratch/<student\_name>/baci}
  \item {After the student made some changes to the code he can commit their work to the \/Baci\_bare in their
home directory. This way they can access their code from another machine next time and the code is backed up by the lrz }	
\end{enumerate}

\paragraph{Workflow from a students point of view}
\begin{enumerate}
  \item Student makes changes in his/her baci
  \item In case the student creates new file, git commit -a is not enough. The new files have to be added first via\\
  \texttt{git add src/newfile1 src/newfile2}
  \item Student commits the changes to his local git repository in /scratch/<student\_name>/workspace/baci\\
  \texttt{git commit -a -m "commit message"}
  \item Transfer changes to baci\_bare in students home so that the supervisor can harvest the changes:\\
  \texttt{git push}
  \item After further changes go back to step 2.
\end{enumerate}


\paragraph{Update baci version of the student}
In order to update the student's code you have to update the svn and the git instances of BACI on your machine by:
  \begin{enumerate}
    \item update BACI in Baci\_stud:\\
    \texttt{svn update}
    \item commit the new version of BACI to the repository (branch master; only the active branch is updated) \\
    Where are you? Which branch is active?: \texttt{git branch}\\
    Switch between branches: \texttt{git checkout master} \\
    Check actual status: \texttt{git status} \\
    add and commit files and modifications into your repository: \texttt{git add .} and \texttt{git commit -a}
    \item this command is necessary to get the latest version of the student's Baci\_bare, which is merged before with the working copy from the student's scratch) \\ 
    \texttt{git checkout baci\_stud} \\
    \texttt{git pull student@red??.stud:$\sim$/Baci\_bare */refs/heads/*:ref/remotes/origin/*} (in folder Baci\_stud on branch student\_name)\\
    $\rightarrow$ possible conflicts need to be resolved (see 'Merge baci version of the student with baci version (svn)' \ref{resolveConflict})
    \item merge branch master (actual BACI version from the svn repository) into the branch student\_name (actual version of your student):\\
    check if you are on branch baci\_stud \\
    \texttt{git merge master} \\
    $\rightarrow$ possible conflicts need to be resolved (see 'Merge baci version of the student with baci version (svn)' \ref{resolveConflict})
    \item \texttt{git push student@red??.stud:$\sim$/Baci\_bare } (in folder Baci\_stud)
    push down the latest revision of Baci (now merged with the students changes) down into his home directory.	
  \end{enumerate}

\paragraph{Merge baci version of the student with baci version (svn)}

I am not completely sure about these steps. I will test it with my next student.

\begin{enumerate}
 \item Merge Baci\_bare with the working copy from the student's scratch
 \item Merge Baci\_bare with Baci\_stud \\
 $ \rightarrow $ \texttt{git fetch student@red??.stud:$\sim$/Baci\_bare */refs/heads/*:ref/remotes/origin/*} (in folder Baci\_stud) \\
 $ \rightarrow $ \texttt{git merge ref/remotes/origin/*:*/refs/heads/*} ?????
 \item no conficts $ \rightarrow $ be happy
 \item conflicts: \label{resolveConflict}
 \begin{itemize}
  \item resolve conflicts with, e.g., kdiff3 \\
  $ \rightarrow $ \texttt{git mergetool -t kdiff3} (maybe you must install kdiff3)
  \item add and commit the files with resolved conflicts \\
  $ \rightarrow $ \texttt{git add 'files with conflicts'} \\
  $ \rightarrow $ \texttt{git commit -a}
  \item congratulation, you merged two versions
 \end{itemize}
 \item update BACI on branch master
 \item merge branch master into branch baci\_stud
 \item commit the actual version into the svn repository
\end{enumerate}

\paragraph{Usefull commands}

\begin{itemize}
  \item \texttt{git status} \\
  overview over the actual status of your repository: untracked files (can be added to the repository) and tracked files with modifications (can be commited)
  \item \texttt{git branch} \\ 
  list all branches in the current clone
  \item  \texttt{git diff \$*$\mid$ kompare -} \\
  compare the files in kompare
  \item \texttt{git diff branch\_name \$*$\mid$ kompare -} \\
  compare current branch to branch\_name in kompare
  \item Before you can update your version in your home directory, you need to commit your modifications on the branch student\_name (scratch). Commands are executed in /scratch/student\_name/workspace/Baci:
  \begin{itemize}
    \item \texttt{git commit -a }, \item \texttt{git commit 'file'}\\
    After commiting the modifications the command 'git diff' does not show anymore any differences in the files. Now, the git version in the branch student\_name on scratch matches the version on your computer 
    \item \texttt{git add .}, \texttt{git add 'file' }\\
    all files or a particular file is added to the version control
  \end{itemize}
  \item At different computers the student has always access to the latest baci version stored in his home directory (Baci\_bare). Therefore, Baci\_bare need to be updated regularly.
  \begin{itemize}
    \item git pull cannot be executed from baci\_bare: \\
  \texttt{ git push origin} (executed student's scratch)
    \item getting latest version from Baci\_bare: \\
   \texttt{ git pull origin} (executed student's scratch) 
  \end{itemize}
  
  \item \texttt{gitk} \\
  graphical representation of the history
\end{itemize}



\subsubsection{Eclipse version: Galileo}

\begin{enumerate}
\item execute \texttt{eclipse} (see paragraph: In students home directory) to start eclipse

\item browse to your \texttt{workspace}: \texttt{/scratch/student\_name/workspace} and press Ok

\item If you are not already there, choose "Workbench"

\item \texttt{File -> New -> C++ Project...}

\item project name need to be identical to the folder \texttt{baci} generated with git
\begin{itemize}
  \item \texttt{Project name: baci} 
  \item choose \texttt{Executable $\rightarrow$ empty project} 
  \item press \texttt{Finish}
  \item your folder structure should show up in the tab \texttt{Project Explorer} 
\end{itemize}

\item Change to the shell and configure BACI as usual:
\begin{itemize}
  \item \texttt{mkdir baci\_debug}
  \item \texttt{cd baci\_debug}
  \item \texttt{../do-configure-debug}
  \item it will take some seconds
\end{itemize}

\item Now go back to eclipse

\item If you are not already there, choose "Workbench".

\item Right-Click on BACI in the Project Explorer and Choose \texttt{Properties}

\item \texttt{C/C++ Build} $ \rightarrow $ tab \texttt{Builder Settings}:
\begin{itemize}
 \item De-Select \texttt{Generate Makefiles automatically}
 \item \texttt{Build directory}: browse via \texttt{File system} to your folder \\
 /scratch/student\_name/workspace/baci/baci\_debug
\end{itemize}

\item \texttt{C/C++ Build} $ \rightarrow $ tab \texttt{Behaviour}:
\begin{itemize}
  \item Activate \texttt{Use parallel build} on 2 processors
  \item remove "all" from \texttt{Build on resource save (Auto build)}
  \item remove "all" from \texttt{Build (Incremental build)}
\end{itemize}

\item Open subitem of \texttt{C/C++ Build} named \texttt{Discovery Options}
\begin{itemize}
 \item \texttt{Discover Profile Options}: browse to your previously created Makefile
 \item press \texttt{Load} to load all defines flag from your Makefile (Baci\_debug/Makefile)
\end{itemize}

\item Open subitem of \texttt{C/C++ General} named \texttt{Path and Symbols}
\begin{itemize}
  \item tab \texttt{Includes}, add: Directory $\rightarrow$ Max and select \texttt{Add to all configurations}, \texttt{Add to all language} (\texttt{Is a workspace path} need to be deselected)
  \item Under \texttt{Symbols}, add: Name $\rightarrow$ Moritz and select \texttt{Add to all configurations}, \texttt{Add to all language}
\end{itemize}

\item Restart eclipse!

\item compile project by \texttt{Project} $\rightarrow$ \texttt{Build all}

\item configure tab size:
\begin{itemize}
 \item window $\rightarrow$ {\it{preferences}}
 \item {\it{C/C++ General}} $\rightarrow$  {\it{Code Style}}
 \item tick {\it{enable project specific settings}}
 \item chose {\it{myprofile}} (until now there is no institute style file) or
 \item generate a new profile including 
  \begin{itemize}
    \item Tab policy: spaces only
    \item indendation size: 2 white spaces 
    \item tab size: 2 white spaces
  \end{itemize}
\end{itemize}

\end{enumerate}

\subsubsection{Debugging in eclipse}

\begin{itemize}

\item Add debug configuration (small bug in the bar): double click on \texttt{C/C++ local application}

\item Configure debug configuration: (identical to command \texttt{./baci-debug (Inputfile) (Outputfile)} in the shell):

\begin{itemize}
\item Under \texttt{Main}
\item choose your project
\item choose your application (Baci\_debug/baci-debug)
\item Under \texttt{Arguments}
\item add \texttt{\$ \{file\_prompt\} XXX} to box
\item choose your working direction (all data are saved in this folder)
\end{itemize}

\end{itemize}

BACI is now checked out and ready to use.

