%% mathematical abbreviations

%% mathematical operators
\newcommand{\ii}{\mathrm{i}}  % complex i
\newcommand{\dd}{\mathrm{d}}  % differential d
\newcommand{\pd}{\partial}  % partial differentiation d
\newcommand{\DD}{\mathrm{D}}  % linearisation operator
%\newcommand{\tns}[1]{\underline{\boldsymbol{#1}}}  % tensor
\newcommand{\tns}[1]{
\dimen1=0.5pt  % truncation length of underline
\mathchoice{
  \setbox0=\hbox{$\boldsymbol{#1}$}
  \dimen2=\wd0
  \ifdim\dimen2>\dimen1
    \advance\dimen2 by -2\dimen1
  \fi
  \wd0=\dimen2
  \setbox1=\hbox{\hskip\dimen1\underline{\hskip-\dimen1\box0}\hskip2\dimen1}
  \box1
}{
  \setbox0=\hbox{$\boldsymbol{\textstyle #1}$}
  \dimen2=\wd0
  \ifdim\dimen2>\dimen1
   \advance\dimen2 by -2\dimen1
  \fi
  \wd0=\dimen2
  \setbox1=\hbox{\hskip\dimen1\underline{\hskip-\dimen1\box0}\hskip2\dimen1}
  \box1
}{
  \setbox0=\hbox{$\boldsymbol{\scriptstyle #1}$}
  \dimen2=\wd0
  \ifdim\dimen2>\dimen1
    \advance\dimen2 by -2\dimen1
  \fi
  \wd0=\dimen2
  \setbox1=\hbox{\hskip\dimen1\underline{\hskip-\dimen1\box0}\hskip2\dimen1}
  \box1
}{
  \setbox0=\hbox{$\boldsymbol{\scriptscriptstyle #1}$}
  \dimen2=\wd0
  \ifdim\dimen2>\dimen1
    \advance\dimen2 by -2\dimen1
  \fi
  \wd0=\dimen2
  \setbox1=\hbox{\hskip\dimen1\underline{\hskip-\dimen1\box0}\hskip2\dimen1}
  \box1
}}
\newcommand{\hattns}[1]{\hat{\tns{#1}}}  % tensor with hat
\newcommand{\dottns}[1]{\dot{\tns{#1}}}  % tensor with dot
\newcommand{\bartns}[1]{\bar{\tns{#1}}}  % tensor with bar
\newcommand{\bretns}[1]{\breve{\tns{#1}}}  % tensor with breve
\newcommand{\chetns}[1]{\check{\tns{#1}}}  % tensor with check
\newcommand{\ddottns}[1]{\ddot{\tns{#1}}}  % tensor with to dots
\newcommand{\mat}[1]{\boldsymbol{#1}}  % matrix
\newcommand{\barmat}[1]{\bar{\mat{#1}}}  % matrix with bar
\newcommand{\vct}[1]{\boldsymbol{#1}}  % vector
\newcommand{\hatvct}[1]{\hat{\vct{#1}}}  % vector with hat
\newcommand{\hatdotvct}[1]{\hat{\dotvct{#1}}}  % vector with hat and dot
\newcommand{\dotvct}[1]{\dot{\vct{#1}}}  % vector with dot
\newcommand{\ddotvct}[1]{\ddot{\vct{#1}}}  % vector with double dots
\newcommand{\dddotvct}[1]{\dddot{\vct{#1}}}  % vector with triple dots
\newcommand{\barvct}[1]{\bar{\vct{#1}}}  % vector with bar
\newcommand{\tildevct}[1]{\tilde{\vct{#1}}}  % vector with tilde
\newcommand{\chevct}[1]{\check{\vct{#1}}}  % vector with check
\newcommand{\T}{\mathrm{T}}  % transpose T
\newcommand{\bre}[1]{\breve{#1}}  % short for breve
\newcommand{\che}[1]{\check{#1}}  % short for check
\newcommand{\grad}{grad}  % (spatial) gradient
\newcommand{\Grad}{Grad}  % (material) gradient
\newcommand{\ddiv}{div}  % divergence in spatial frame
\newcommand{\Ddiv}{Div}  % divergence in material frame
\newcommand{\dirac}{\delta}  % Dirac's delta
\newcommand{\cof}{cof}  % (spatial) gradient
\newcommand{\forwhich}{\;|\;}
% \newcommand{\ass}[3][d]{%  % assembly operator, 'd' display style, 't' text style
% \if#1d
% \overset{#3}{\underset{#2}{\raisebox{-0.6ex}{\mbox{\huge $\mathsf{A}$}}}}
% \else
% \raisebox{-0.4ex}{\mbox{\Large $\mathsf{A}$}}_{#2}^{#3}
% \fi
% }
\newcommand{\Ass}[3][d]{%  % assembly operator
% option 'd' is useless
\mathchoice{  % display style
\overset{#3}{\underset{#2}{\raisebox{-0.6ex}{\mbox{\huge $\mathsf{A}$}}}}
}{  % text style
\raisebox{-0.35ex}{\mbox{\Large $\mathsf{A}$}}_{#2}^{#3}
}{  % script style
\raisebox{-0.35ex}{\mbox{\Large $\mathsf{A}$}}_{#2}^{#3}
}{  % scriptscript style
\raisebox{-0.35ex}{\mbox{\Large $\mathsf{A}$}}_{#2}^{#3}
}}

\newcommand{\loc}{loc}  % location mapping
\newcommand{\stat}{stat}  % location mapping
\newcommand{\domainsum}{\bigcup}
\newcommand{\abs}[1]{|#1|}
\newcommand{\Abs}[1]{\|#1\|}
% \ifmmode ... \else ... \fi
\newcommand{\Nabla}{\nabla_{\!0}}
\newcommand{\incr}{\Delta}
\newcommand{\Lin}{Lin}  % linearisation
\newcommand{\trace}{tr}  % trace of quadratic tensor

%% symbols
\newcommand{\eps}{\varepsilon}  % epsilon
\newcommand{\sig}{\sigma}  % epsilon
\newcommand{\const}{\mathrm{const}}  % constant
\newcommand{\R}{\mathbb{R}}  % real set of numbers R
\newcommand{\C}{\mathbb{C}}  % complex set of numbers C
\newcommand{\ndof}{\mathit{ndof}}  % number of degrees of freedom
\newcommand{\nnod}{\mathit{nnod}}  % number of nodes
\newcommand{\nele}{\mathit{nele}}  % number of elements
\newcommand{\enod}{\mathit{enod}}  % number of element nodes
\newcommand{\numgp}{\mathit{numgp}}  % number of Gausspoints in direction 
\newcommand{\nco}{\mathit{nco}}  % number of active contact sets 
\newcommand{\BE}{\text{BE}}  % balance equations
\newcommand{\FBC}{\text{FBC}}  % flux boundary conditions
\newcommand{\DBC}{\text{DBC}}  % Dirichlet boundary conditions
\newcommand{\TPE}{\text{TPE}}  % total potential energy
\newcommand{\EEE}{\text{EE}}  % strain-strain connection
\newcommand{\SSS}{\text{SS}}  % stress-stress connection
\newcommand{\HR}{\text{HR}}  % Hellinger--Reissner
\newcommand{\HW}{\text{HW}}  % Hu--Washizu
\newcommand{\Int}{\text{int}}  % internal
\newcommand{\Ext}{\text{ext}}  % external
\newcommand{\HOT}{\mathit{HOT}}  % higher order terms
\newcommand{\Dt}{\Delta t}  % time step size
\newcommand{\effdyn}{\text{effdyn}}  % effective dynamic
\newcommand{\PK}{\text{PK2}}  % 2nd Piola--Kirchhoff
\newcommand{\Pk}{\text{PK1}}  % 1st Piola--Kirchhoff
\newcommand{\KI}{\text{K}}  % Kirchhoff
\newcommand{\GL}{\text{GL}}  % Green--Lagrange
\newcommand{\EA}{\text{E}}  % Euler--Almansi
\newcommand{\Eng}{\text{i}}  % engineering
\newcommand{\Nat}{\text{n}}  % natural / logarithmic
\newcommand{\TSO}{\textrm{II}}  % theory of second order
\newcommand{\TFO}{\textrm{I}} % theory of 1st order
\newcommand{\CT}{\text{C}}  % Cuachy / true
\newcommand{\BI}{\text{B}}  % Biot
\newcommand{\Tang}{\text{T}}  % tangential
\newcommand{\ela}{\text{e}}  % elastic
\newcommand{\geo}{\text{g}}  % geometric
\newcommand{\ini}{\text{u}}  % initial
\newcommand{\loa}{\text{L}}  % load
\newcommand{\tol}{\mathit{tol}}


%% convenience
\newcommand{\unit}[1]{\,\mathrm{#1}}  % unit
\newcommand{\EE}[1]{\cdot 10^{#1}}  % "scientific notation" for 10 to the power of 
\newcommand{\virt}{\delta}  % variation of first kind
\newcommand{\set}[1]{\mathcal{#1}}  % set
\newcommand{\script}[1]{\mathcal{#1}}

%% text font stuff
\newcommand{\period}{\text{.}}
\newcommand{\comma}{\text{,}}
%\newcommand{\colon}{\text{:}}
\newcommand{\semicolon}{\text{;}}

%% special
\newcommand{\sboxed}[1]{\boxed{\scriptstyle#1}}
\newcommand{\Underline}[1]{\underline{\underline{#1}}}  % double underline
\newcommand{\underneath}[3][0pt]{  % place text underneath symbol
\setbox0=\hbox{$|$}
\dimen2=\ht0
\advance\dimen2 by #1
\setbox0=\hbox{\vrule height\dimen2}
\setbox1=\hbox{$\displaystyle #2$}
\setbox5=\hbox{$#3$}
\dimen5=\dp5
\advance\dimen5 by 2pt
\setbox2=\hbox{\vrule width0pt height\ht5 depth\dimen5 \box5}
\dimen0=\dp0
\advance\dimen0 by \ht2
\advance\dimen0 by 2pt
\dimen1=0.5\wd1
\ifdim\wd1>\wd2
\dimen4=0.5\wd2
\else
\dimen4=0.5\wd1
\fi
\setbox3=\hbox{\copy0\hskip-\dimen4\lower\dimen0\copy2}
\dimen3=\ht3
\advance\dimen3 by \dp1
\advance\dimen3 by 2pt
\setbox4=\hbox to \wd1 {\copy1\hskip-\dimen1\lower\dimen3\copy3}
\box4
}
\newcommand{\overneath}[3][0pt]{  % place text underneath symbol
\setbox0=\hbox{$|$}
\dimen2=\ht0
\advance\dimen2 by #1
\setbox0=\hbox{\vrule height\dimen2}
\setbox1=\hbox{$\displaystyle #2$}
\setbox2=\hbox{$#3$}
\dimen0=\ht0
\advance\dimen0 by \dp2
\advance\dimen0 by 2pt
\dimen1=0.5\wd1
\ifdim\wd1>\wd2
\dimen4=0.5\wd2
\else
\dimen4=0.5\wd1
\fi
\setbox3=\hbox{\copy0\hskip-\dimen4\raise\dimen0\copy2}
\dimen3=\ht1
%\advance\dimen3 by \ht1
\advance\dimen3 by 2pt
\setbox4=\hbox to \wd1 {\copy1\hskip-\dimen1\raise\dimen3\copy3}
\box4
}
%%
%%
\makeatletter
\newcommand{\strikethrough}[2][d]{%
\unitlength=1pt%
\newcount\extraspace
\extraspace=5
\newbox\arg%
% switch to (displayed) math mode
\ifmmode  % switch to (displayed) math mode or text mode
\if#1d  % switch to display style if 'd' option
\setbox\arg=\hbox{$\displaystyle #2$}%
\else  % switch to text style  if 't' option
\setbox\arg=\hbox{$\textstyle #2$}%
\fi
\else
\setbox\arg=\hbox{#2}%
\fi
% get argdepth
\newcount\argdepth%
\setbox1=\hbox{%
\global\argdepth=\strip@pt\dp\arg
\global\advance\argdepth by \the\extraspace
}%
% get argheight
\newcount\argheight%
\setbox1=\hbox{%
\global\argheight=\strip@pt\ht\arg
\global\advance \argheight by \the\argdepth
%\global\advance \argheight by \the\extraspace
}%
% get argwidth
\newcount\argwidth%
\setbox1=\hbox{%
\global\argwidth=\strip@pt\wd\arg
\global\advance\argwidth by \the\extraspace
\global\advance\argwidth by \the\extraspace
}%
% create output box
\newbox\product%
\setbox\product=\hbox{%
\begin{picture}(\the\argwidth,\the\argheight)
% draw a nice box (debug purposes)
%\put(0,-\the\argdepth){\line(1,0){\the\argwidth}}
%\put(0,-\the\argdepth){\line(0,1){\the\argheight}}
%\put(0,-\the\argdepth){\line(0,1){\the\argheight}}
%\put(0,\strip@pt\ht\arg){\line(1,0){\the\argwidth}}
\qbezier(0,-\the\argdepth)(0,-\the\argdepth)(\the\argwidth,\the\argheight)
%\put(\the\argwidth,\the\argheight){tip}  % debug purposes
\put(\the\extraspace,0){\box\arg}
\end{picture}}%
\box\product%
}
\makeatother

%%% Local Variables: 
%%% mode: plain-tex
%%% TeX-master: "fe"
%%% End: 
