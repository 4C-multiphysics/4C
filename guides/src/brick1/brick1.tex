
%%
%%::::::::::::::::::::::::::::commands
% font
%\usepackage{helvet}
%\renewcommand{\familydefault}{\sfdefault}
% sectioning
%\makeatletter

\newcommand{\subpart}[1]{%
  \vspace{2\parskip}%
  \textbf{#1}%
  \hspace*{1em}%
}
\newcommand{\subsubpart}[1]{%
  \vspace{2\parskip}%
  \textit{#1}%
  \hspace*{1em}%
}
\newcommand{\subitempart}[2][$\bullet$]{%
  \vspace{2\parskip}%
  #1\hspace{1.em}\textbf{#2}\hspace{1.em}%
}
\newcommand{\subsubitempart}[2][--]{%
  \vspace{2\parskip}%
  #1\hspace{1.em}\textit{#2}\hspace{1.em}%
}
\newcommand{\note}[2][]{%
  \textit{#2}\hspace*{1em}
}
\makeatletter
\newcommand\subxsection{\@startsection{subsection}{2}{\z@}%
                                     {-3.25ex\@plus -1ex \@minus -.2ex}%
                                     {1.5ex \@plus .2ex}%
                                     {\normalfont\large\bfseries}}
\makeatother
%\makeatother
% miscellaneous
\newenvironment{inframe}{%
\noindent
\begin{lrbox}{\setbox1}
\begin{minipage}{\linewidth}%
}{%
\end{minipage}
\end{lrbox}
\framebox{\usebox{\box1}}
}
% foot note with choice of symbol 
\long\def\symbolfootnote[#1]#2{\begingroup%
\def\thefootnote{\fnsymbol{footnote}}\footnote[#1]{#2}\endgroup}
\long\def\symbolfootnotetext[#1]#2{\begingroup%
\def\thefootnote{\fnsymbol{footnote}}\footnotetext[#1]{#2}\endgroup}
% convenience
\newcommand{\ger}[1]{\emph{#1}}
\newcommand{\lat}[1]{#1}
\newcommand{\ie}{i.e.\@}
\newcommand{\cf}{cf.\@}
\newcommand{\eg}{e.g.\@}
\newcommand{\etal}{et al.\@}
\newcommand{\intro}[1]{\emph{#1}}
\newcommand{\fail}[1]{\emph{#1}}
%\newcommand{\tc}[1]{\ovalfbox{\textsf{#1}}}
\newcommand{\tc}[1]{%
\ifmmode
  \mathchoice{
    \ovalfbox{\textsf{\text{$#1$}}}
  }{
    \ovalfbox{\textsf{\text{$#1$}}}
  }{
    \ovalfbox{\textsf{\text{$\scriptstyle #1$}}}
  }{
    \ovalfbox{\textsf{\text{$\scriptscriptstyle #1$}}}
  }
\else
\ovalfbox{\textsf{#1}}%
\fi}
\newcommand{\name}[1]{#1}
% rotate
\newcommand{\textrotl}[1]{%
\setbox2=\hbox{#1}
\rotl2
}
% source code style
\newcommand{\cod}[1]{\texttt{\textup{#1}}}
\newenvironment{codenv}{\begin{quote} \tt}{\end{quote}}
\newcommand{\cor}{$|$}
\newcommand{\cgb}{$($}
\newcommand{\cge}{$)$}
\newcommand{\ccb}{$[$}
\newcommand{\cce}{$]$}
\newcommand{\cnl}{} %%$\setminus$}
\newcommand{\chs}{\hspace{3em}}


% bold Figure 7.5 at figures
\makeatletter
\long\def\@makecaption#1#2{%
  \vskip-0.65\abovecaptionskip
  \sbox\@tempboxa{\textbf{#1:} #2}%
  \ifdim \wd\@tempboxa >\hsize
    \textbf{#1:} #2\par
  \else
    \global \@minipagefalse
    \hb@xt@\hsize{\hfil\box\@tempboxa\hfil}%
  \fi
  \vskip\belowcaptionskip}
\makeatother
%%% Local Variables: 
%%% mode: latex
%%% TeX-master: "inventory"
%%% End: 

%% mathematical abbreviations
%%\usepackage{helvet}
%%\renewcommand{\familydefault}{\sfdefault}
%\usepackage{amsmath,amssymb,color}  % and more

% extra colourful stuff
%%\newcommand{\mathbl}[1]{\text{\textcolor[rgb]{0.0,0.0,0.69}{$#1$}}}
%\newcommand{\mathbl}[1]{#1}
%%\newcommand{\mathgr}[1]{\text{\textcolor[rgb]{0.0,0.69,0.0}{$#1$}}}
%\newcommand{\mathgr}[1]{#1}

%% mathematical operators
\newcommand{\ii}{\mathrm{i}}  % complex i
\newcommand{\dd}{\mathrm{d}}  % differential d
\newcommand{\pd}{\partial}  % partial differentiation d
\newcommand{\DD}{\mathrm{D}}  % linearisation operator
\newcommand{\Alpha}{\mathcal{A}}
\newcommand{\levicivita}{\mathcal{E}}  % Levi-Civita symbol
%\newcommand{\tns}[1]{\underline{\boldsymbol{#1}}}  % tensor
%\newcommand{\tnsfour}[1]{\mathsf{#1}}
\newcommand{\tns}[2][0]{
\def\myinput{#2}
% \count0=#1
% \ifnum\count0=1
%   \def\myinput{#2}
% \fi
% \ifnum\count0=2
%   \def\myinput{#2}
% \fi
% \ifnum\count0=4
%   \def\myinput{\tnsfour{#2}}
% \fi
\dimen1=0.5pt  % truncation length of underline
\mathchoice{
  \setbox0=\hbox{$\boldsymbol{\myinput{}}$}
  \dimen2=\wd0
  \ifdim\dimen2>\dimen1
    \advance\dimen2 by -2\dimen1
  \fi
  \wd0=\dimen2
  \setbox1=\hbox{\hskip\dimen1\underline{\hskip-\dimen1\box0}\hskip2\dimen1}
  \box1
}{
  \setbox0=\hbox{$\boldsymbol{\textstyle \myinput{}}$}
  \dimen2=\wd0
  \ifdim\dimen2>\dimen1
   \advance\dimen2 by -2\dimen1
  \fi
  \wd0=\dimen2
  \setbox1=\hbox{\hskip\dimen1\underline{\hskip-\dimen1\box0}\hskip2\dimen1}
  \box1
}{
  \setbox0=\hbox{$\boldsymbol{\scriptstyle \myinput{}}$}
  \dimen2=\wd0
  \ifdim\dimen2>\dimen1
    \advance\dimen2 by -2\dimen1
  \fi
  \wd0=\dimen2
  \setbox1=\hbox{\hskip\dimen1\underline{\hskip-\dimen1\box0}\hskip2\dimen1}
  \box1
}{
  \setbox0=\hbox{$\boldsymbol{\scriptscriptstyle \myinput{}}$}
  \dimen2=\wd0
  \ifdim\dimen2>\dimen1
    \advance\dimen2 by -2\dimen1
  \fi
  \wd0=\dimen2
  \setbox1=\hbox{\hskip\dimen1\underline{\hskip-\dimen1\box0}\hskip2\dimen1}
  \box1
}}
\newcommand{\hattns}[1]{\hat{\tns{#1}}}  % tensor with hat
\newcommand{\dottns}[1]{\dot{\tns{#1}}}  % tensor with dot
\newcommand{\bartns}[1]{\bar{\tns{#1}}}  % tensor with bar
\newcommand{\bretns}[1]{\breve{\tns{#1}}}  % tensor with breve
\newcommand{\chetns}[1]{\check{\tns{#1}}}  % tensor with check
\newcommand{\tiltns}[1]{\tilde{\tns{#1}}}  % tensor with tilde
\newcommand{\ddottns}[1]{\ddot{\tns{#1}}}  % tensor with to dots
\newcommand{\mat}[1]{\boldsymbol{#1}}  % matrix
\newcommand{\barmat}[1]{\bar{\mat{#1}}}  % matrix with bar
\newcommand{\vct}[1]{\boldsymbol{#1}}  % vector
\newcommand{\hatvct}[1]{\hat{\vct{#1}}}  % vector with hat
\newcommand{\hatdotvct}[1]{\hat{\dotvct{#1}}}  % vector with hat and dot
\newcommand{\dotvct}[1]{\dot{\vct{#1}}}  % vector with dot
\newcommand{\ddotvct}[1]{\ddot{\vct{#1}}}  % vector with double dots
\newcommand{\dddotvct}[1]{\dddot{\vct{#1}}}  % vector with triple dots
\newcommand{\barvct}[1]{\bar{\vct{#1}}}  % vector with bar
\newcommand{\tildevct}[1]{\tilde{\vct{#1}}}  % vector with tilde
\newcommand{\tilvct}[1]{\tilde{\vct{#1}}}  % vector with tilde
\newcommand{\chevct}[1]{\check{\vct{#1}}}  % vector with check
\newcommand{\T}{\mathrm{T}}  % transpose T
\newcommand{\bre}[1]{\breve{#1}}  % short for breve
\newcommand{\che}[1]{\check{#1}}  % short for check
\newcommand{\adiag}[1]{\overset{\scriptscriptstyle\smallsetminus}{#1}}
\newcommand{\alowtri}[1]{\overset{\scriptscriptstyle\llcorner}{#1}}
\newcommand{\grad}{\mathrm{grad}}  % (spatial) gradient
\newcommand{\Grad}{\mathrm{Grad}}  % (material) gradient
\newcommand{\ddiv}{\mathrm{div}}  % divergence in spatial frame
\newcommand{\Ddiv}{\mathrm{Div}}  % divergence in material frame
\newcommand{\dirac}{\delta}  % Dirac's delta
\newcommand{\cof}{\mathrm{cof}}  % (spatial) gradient
\newcommand{\forwhich}{\;|\;}
% \newcommand{\ass}[3][d]{%  % assembly operator, 'd' display style, 't' text style
% \if#1d
% \overset{#3}{\underset{#2}{\raisebox{-0.6ex}{\mbox{\huge $\mathsf{A}$}}}}
% \else
% \raisebox{-0.4ex}{\mbox{\Large $\mathsf{A}$}}_{#2}^{#3}
% \fi
% }
\newcommand{\Ass}[3][d]{%  % assembly operator
% option 'd' is useless
\mathchoice{  % display style
\overset{#3}{\underset{#2}{\raisebox{-0.6ex}{\mbox{\huge $\mathsf{A}$}}}}
}{  % text style
\raisebox{-0.35ex}{\mbox{\Large $\mathsf{A}$}}_{#2}^{#3}
}{  % script style
\raisebox{-0.35ex}{\mbox{\Large $\mathsf{A}$}}_{#2}^{#3}
}{  % scriptscript style
\raisebox{-0.35ex}{\mbox{\Large $\mathsf{A}$}}_{#2}^{#3}
}}

\newcommand{\loc}{\mathrm{loc}}  % location mapping
\newcommand{\stat}{\mathrm{stat}}  % location mapping
\newcommand{\domainsum}{\bigcup}
\newcommand{\abs}[1]{|#1|}
\newcommand{\Abs}[1]{\|#1\|}
% \ifmmode ... \else ... \fi
\newcommand{\Nabla}{\nabla_{\!0}}
\newcommand{\incr}{\Delta}
\newcommand{\Lin}{\mathrm{Lin}}  % linearisation
\newcommand{\trace}{\mathrm{tr}}  % trace of quadratic tensor

%% symbols
\newcommand{\eps}{\varepsilon}  % epsilon
\newcommand{\sig}{\sigma}  % epsilon
\newcommand{\const}{\mathrm{const}}  % constant
\newcommand{\R}{\mathbb{R}}  % real set of numbers R
\newcommand{\CC}{\mathbb{C}}  % complex set of numbers C
\newcommand{\ndof}{\mathit{ndof}}  % number of degrees of freedom
\newcommand{\nnod}{\mathit{nnod}}  % number of nodes
\newcommand{\nele}{\mathit{nele}}  % number of elements
\newcommand{\eenh}{\mathit{eenh}}  % number of element enhancements
\newcommand{\enod}{\mathit{enod}}  % number of element nodes
\newcommand{\numgp}{\mathit{numgp}}  % number of Gausspoints in direction 
\newcommand{\nco}{\mathit{nco}}  % number of active contact sets 
\newcommand{\BE}{\text{BE}}  % balance equations
\newcommand{\FBC}{\text{FBC}}  % flux boundary conditions
\newcommand{\DBC}{\text{DBC}}  % Dirichlet boundary conditions
\newcommand{\TPE}{\text{TPE}}  % total potential energy
\newcommand{\EEE}{\text{EE}}  % strain-strain connection
\newcommand{\SSS}{\text{SS}}  % stress-stress connection
\newcommand{\FFF}{\text{FF}}  % defgrad-defgard connection
\newcommand{\PPP}{\text{PP}}  % 1PK-1PK stress connection 
\newcommand{\HR}{\text{HR}}  % Hellinger--Reissner
\newcommand{\HW}{\text{HW}}  % Hu--Washizu
\newcommand{\Int}{\text{int}}  % internal
\newcommand{\Ext}{\text{ext}}  % external
\newcommand{\HOT}{\mathit{HOT}}  % higher order terms
\newcommand{\Dt}{\Delta t}  % time step size
\newcommand{\effdyn}{\text{effdyn}}  % effective dynamic
\newcommand{\Teffdyn}{\text{Teffdyn}}  % effective dynamic
\newcommand{\PK}{\text{PK2}}  % 2nd Piola--Kirchhoff
\newcommand{\Pk}{\text{PK1}}  % 1st Piola--Kirchhoff
\newcommand{\KI}{\text{K}}  % Kirchhoff
\newcommand{\GL}{\text{GL}}  % Green--Lagrange
\newcommand{\EA}{\text{E}}  % Euler--Almansi
\newcommand{\Eng}{\text{i}}  % engineering
\newcommand{\Nat}{\text{n}}  % natural / logarithmic
\newcommand{\TSO}{\textrm{II}}  % theory of second order
\newcommand{\TFO}{\textrm{I}} % theory of 1st order
\newcommand{\CT}{\text{C}}  % Cauchy / true
\newcommand{\BI}{\text{B}}  % Biot
\newcommand{\VK}{\text{VK}}  % St. Venant-Kirchhoff
\newcommand{\Tang}{\text{T}}  % tangential
\newcommand{\ela}{\text{e}}  % elastic
\newcommand{\geo}{\text{g}}  % geometric
\newcommand{\ini}{\text{u}}  % initial
\newcommand{\loa}{\text{L}}  % load
\newcommand{\tol}{\mathit{tol}}  % tolerance
\newcommand{\LL}{\text{L}}  % `linear'
\newcommand{\enh}{\mathit{enh}}  % enhanced
\newcommand{\red}{\text{red}}  % reduced
\newcommand{\crit}{\text{crit}}  % critical
\newcommand{\Inva}{I}
\newcommand{\INva}{I\!\!I}
\newcommand{\INVa}{I\!\!I\!\!I}
\newcommand{\Mat}{\text{m}}


%% convenience
\newcommand{\unit}[1]{\,\mathrm{#1}}  % unit
\newcommand{\EE}[1]{\cdot 10^{#1}}  % "scientific notation" for 10 to the power of 
\newcommand{\virt}{\delta}  % variation of first kind / virtual
\newcommand{\set}[1]{\mathcal{#1}}  % set
\newcommand{\script}[1]{\mathcal{#1}}

%% text font stuff
\newcommand{\period}{\text{.}}
\newcommand{\comma}{\text{,}}
%\newcommand{\colon}{\text{:}}
\newcommand{\semicolon}{\text{;}}

%% circled text (EgyptTeX)
%%\newcommand{\tc}[1]{%
%%\ifmmode
%%  \mathchoice{
%%    \ovalfbox{\textsf{\text{$#1$}}}
%%  }{
%%    \ovalfbox{\textsf{\text{$#1$}}}
%%  }{
%%    \ovalfbox{\textsf{\text{$\scriptstyle #1$}}}
%%  }{
%%    \ovalfbox{\textsf{\text{$\scriptscriptstyle #1$}}}
%%  }
%%\else
%%\ovalfbox{\textsf{#1}}%
%%\fi}

%% special
\newcommand{\sboxed}[1]{\boxed{\scriptstyle#1}}
\newcommand{\Underline}[1]{\underline{\underline{#1}}}  % double underline
\newcommand{\underneath}[3][0pt]{  % place text underneath symbol
\setbox0=\hbox{$|$}
\dimen2=\ht0
\advance\dimen2 by #1
\setbox0=\hbox{\vrule height\dimen2}
\setbox1=\hbox{$\displaystyle #2$}
\setbox5=\hbox{$#3$}
\dimen5=\dp5
\advance\dimen5 by 2pt
\setbox2=\hbox{\vrule width0pt height\ht5 depth\dimen5 \box5}
\dimen0=\dp0
\advance\dimen0 by \ht2
\advance\dimen0 by 2pt
\dimen1=0.5\wd1
\ifdim\wd1>\wd2
\dimen4=0.5\wd2
\else
\dimen4=0.5\wd1
\fi
\setbox3=\hbox{\copy0\hskip-\dimen4\lower\dimen0\copy2}
\dimen3=\ht3
\advance\dimen3 by \dp1
\advance\dimen3 by 2pt
\setbox4=\hbox to \wd1 {\copy1\hskip-\dimen1\lower\dimen3\copy3}
\box4
}
\newcommand{\overneath}[3][0pt]{  % place text underneath symbol
\setbox0=\hbox{$|$}
\dimen2=\ht0
\advance\dimen2 by #1
\setbox0=\hbox{\vrule height\dimen2}
\setbox1=\hbox{$\displaystyle #2$}
\setbox2=\hbox{$#3$}
\dimen0=\ht0
\advance\dimen0 by \dp2
\advance\dimen0 by 2pt
\dimen1=0.5\wd1
\ifdim\wd1>\wd2
\dimen4=0.5\wd2
\else
\dimen4=0.5\wd1
\fi
\setbox3=\hbox{\copy0\hskip-\dimen4\raise\dimen0\copy2}
\dimen3=\ht1
%\advance\dimen3 by \ht1
\advance\dimen3 by 2pt
\setbox4=\hbox to \wd1 {\copy1\hskip-\dimen1\raise\dimen3\copy3}
\box4
}
%%
%%
\makeatletter
\newcommand{\strikethrough}[2][d]{%
\unitlength=1pt%
\newcount\extraspace
\extraspace=5
\newbox\arg%
% switch to (displayed) math mode
\ifmmode  % switch to (displayed) math mode or text mode
\if#1d  % switch to display style if 'd' option
\setbox\arg=\hbox{$\displaystyle #2$}%
\else  % switch to text style  if 't' option
\setbox\arg=\hbox{$\textstyle #2$}%
\fi
\else
\setbox\arg=\hbox{#2}%
\fi
% get argdepth
\newcount\argdepth%
\setbox1=\hbox{%
\global\argdepth=\strip@pt\dp\arg
\global\advance\argdepth by \the\extraspace
}%
% get argheight
\newcount\argheight%
\setbox1=\hbox{%
\global\argheight=\strip@pt\ht\arg
\global\advance \argheight by \the\argdepth
%\global\advance \argheight by \the\extraspace
}%
% get argwidth
\newcount\argwidth%
\setbox1=\hbox{%
\global\argwidth=\strip@pt\wd\arg
\global\advance\argwidth by \the\extraspace
\global\advance\argwidth by \the\extraspace
}%
% create output box
\newbox\product%
\setbox\product=\hbox{%
\begin{picture}(\the\argwidth,\the\argheight)
% draw a nice box (debug purposes)
%\put(0,-\the\argdepth){\line(1,0){\the\argwidth}}
%\put(0,-\the\argdepth){\line(0,1){\the\argheight}}
%\put(0,-\the\argdepth){\line(0,1){\the\argheight}}
%\put(0,\strip@pt\ht\arg){\line(1,0){\the\argwidth}}
\qbezier(0,-\the\argdepth)(0,-\the\argdepth)(\the\argwidth,\the\argheight)
%\put(\the\argwidth,\the\argheight){tip}  % debug purposes
\put(\the\extraspace,0){\box\arg}
\end{picture}}%
\box\product%
}
\makeatother

%%% Local Variables: 
%%% mode: plain-tex
%%% TeX-master: "fe"
%%% End: 


%%
%%::::::::::::::::::::::::::::BRICK1
\chapter{BRICK1}
\begin{flushright}
\emph{``Mr.\@ President, tear down this BRICK1 element!''}\\
\emph{``Et tu, BRICK1?''}\\
\emph{``Ceterum censeo saxum BRICK1 esse delendum!''}
\end{flushright}

\subitempart{About}\\

Input sequences are denoted by applying EBNF, see Sec.~\ref{brick1:sec:ebnf}.

\subitempart{Text input} `\cnl' is for line continuation
\begin{quote}
\cod{BRICK1} \cnl \\
\cgb \cod{HEX8} \cor \cod{HEX20} \cge $int_1$ $\ldots$ $int_\enod$ \cnl \\
\cod{MAT} $int_1$ \cnl \\
\cod{GP} $int_1$ $int_2$ $int_3$ \cnl \\
\cod{GP\_TET} $int_1$ \cnl \chs Gauss points tetrahedra (not read in) \\
\cod{HYP} \cgb \cod{0} \cor \cod{3} \cor \cod{6} \cor \cod{9} \cge \cnl \\
\cod{FORM} \cgb \cod{0} \cor \cod{1} \cor \cod{2} \cge \cnl \\
\cod{STRESSES} \cgb \cod{NoBStr} \cor \cod{GPXYZ} \cor \cod{GPRST} \cor \cod{GP123} \cor
\cod{NDXYZ} \cor \cod{NDRST} \cor \cod{ND123} \cor \cod{NDEQS} \cge \cnl \\
\ccb \cod{TSI\_COUPTYP} \cgb \cod{None} \cor \cod{Thermconf} \cor
\cod{Thermcreate} \cge \cce
\end{quote}

\subitempart{Discretisation types} hex8, hex20

Hex27 fails, shape function derivatives are not set.

Tetrahedron tet4, tet10 can be read in, however, the element routine cannot
build any tetrahedron quantities.

\subitempart{Materials} The material \cod{MAT} index can refer back to
\begin{description}
\item[\cod{MAT\_Struct\_StVenantKirchhoff}] linear elastic
\item[\cod{MAT\_Struct\_STVENPOR}] porous linear elastic
\item[\cod{MAT\_MFOC}] open cell metal foam linear elastic
\item[\cod{MAT\_MFC}] closed cell metal foam linear elastic
\item[\cod{MAT\_NeoHMFCC}] foam, closed cell, based on modified Neo Hook
\item[\cod{MAT\_Struct\_NeoHooke}] kompressible neo-hooke
\item[\cod{MAT\_MisesPlastic}] von mises material law
\item[\cod{MAT\_Struct\_Orthotropic}] elastic orthotropic material law
\item[\cod{m\_pl\_mises\_ls}] \fail{No input!} von mises material law - large
  strains
\item[\cod{MAT\_HYPER\_POLYCONVEX}] 
\end{description}

\subitempart{Gauss points}
\cod{GP} $int_1$ $int_2$ $int_3$ set the number of Gauss--Legendre points for
the $3$ space dimenions; for each $int_1, int_2, int_3 \in [1,6]$

\subitempart{Single- or multi-field approach}
\begin{description}
\item[\cod{HYB 0}] purely displacement-based (works, obviously)
\item[\cod{HYB \cgb 3 \cor 6 \cor 9 \cge}] \fail{EAS untested --- works possibly}
\end{description}

\subitempart{Spatial kinematic}
\begin{description}
\item[\cod{FORM 0}] Linear
\item[\cod{FORM 1}] Updated Lagrangean formulation \fail{Not implemented}
\item[\cod{FORM 2}] Total Lagrangean formulation
\end{description}

\subitempart{Stresses}
\begin{description}
\item[\cod{STRESSES NoBStr}] No stresses
\item[\cod{STRESSES GPXYZ}] Stress $XYZ$-oriented at Gauss points
\item[\cod{STRESSES GPRST}] Stress $rst$-oriented at Gauss points
\item[\cod{STRESSES GP123}] Principal stresses at Gauss points
\item[\cod{STRESSES NDXYZ}] Stress $XYZ$-oriented at element nodes
\item[\cod{STRESSES NDRST}] Stress $rst$-oriented at element nodes
\item[\cod{STRESSES ND123}] Principal stresses at element nodes
\item[\cod{STRESSES NDEQS}] Equivalent stresses at element nodes
\end{description}
The $XYZ$-frame is the material frame. The $rst$-frame is the parameter
frame.
\begin{enumerate}
\item All stress choices work for \cod{xxx.out} output
\item $XYZ$- and $rst$-oriented and equivalent stresses at nodes work for Gid
  \cod{xxx.flavia.res} output
\item $XYZ$-oriented stresses at Gauss points work for Gid
  \cod{xxx.flavia.res} output
\end{enumerate}

\subitempart{Thermo-structure-interaction (TSI)} Check WALL1 section

\subitempart{Neumann BC}\\
On nodes
\begin{quote}
%%\cod{-----------------------------------DESIGN POINT NEUMANN CONDITIONS}\\
\cod{E $int$} \cnl \chs Index of design entity\\
\cod{-} \cnl\\
\cod{\cgb none \cor $int$ \cge} \chs Curve index\\
\cod{\cgb 0 \cor 1 \cge 
     \cgb 0 \cor 1 \cge
     \cgb 0 \cor 1 \cge
     0 0 0} \cnl \chs Flag on/off condition \\
\cod{$real_1$ $real_2$ $real_3$ 0.0 0.0 0.0} \cnl \chs Value\\
\cod{Mid} \chs Reference surface
\end{quote}
\begin{enumerate}
\item Load vectors work.
\item Individual load curves are ignored : do not work.
\end{enumerate}

On lines
\begin{quote}
%%\cod{------------------------------------DESIGN LINE NEUMANN CONDITIONS}\\
\cod{E $int$} \cnl \chs Index of design entity\\
\cod{-} \cnl\\
\cod{\cgb none \cor $int$ \cge} \cnl \chs Curve index\\
\cod{\cgb 0 \cor 1 \cge 
     \cgb 0 \cor 1 \cge 
     \cgb 0 \cor 1 \cge
     0 0 0} \cnl \chs Flag on/off condition \\
\cod{$real_1$ $real_2$ $real_3$ 0.0 0.0 0.0} \cnl \chs Value\\
\cod{\cgb Live \cor Dead \cge} \cnl \chs Load type\\
\cod{Mid} \chs Reference surface
\end{quote}
\begin{enumerate}
\item Constant material loads work.
\item There is no difference between \cod{Live} and \cod{Dead}. All load types
refer to the undeformed/material frame.
\item Individual load curve is not applied.
\end{enumerate}

On surfaces
\begin{quote}
%%\cod{------------------------------------DESIGN SURF NEUMANN CONDITIONS}\\
\cod{E $int$} \cnl \chs Index of design entity\\
\cod{-} \cnl\\
\cod{\cgb none \cor $int$ \cge} \chs Curve\\
\cod{\cgb 0 \cor 1 \cge
     \cgb 0 \cor 1 \cge
     \cgb 0 \cor 1 \cge
     0 0 0} \cnl \chs Flag on/off condition \\
\cod{$real_1$ $real_2$ $real_3$ 0.0 0.0 0.0} \cnl \chs Value (without curve?)\\
\cod{\cgb Live \cor Dead \cor PrescribedDomainLoad \cor orthopressure \cge} \cnl \chs Load type\\
\cod{Mid} \chs Reference surface
\end{quote}
\begin{enumerate}
\item  There is no difference between \cod{Live}, \cod{Dead} and
  \cod{PrescribedDomainLoad}. All load types refer to the undeformed/material
  frame. 
\item Constant domain load referred to material frame works.
\item Load type \cod{orthopressure}, value must be $real_3$. Details
  $\to$ Lena. 
\item Individual load curve is ignored : does not work.
\item Reference surface is meaningless.
\end{enumerate}

On volume
\begin{quote}
%%\cod{-------------------------------------DESIGN VOL NEUMANN CONDITIONS}\\
\cod{E $int$} \cnl \chs Index of design entity\\
\cod{-} \cnl\\
\cod{\cgb none \cor $int$ \cge} \chs Curve\\
\cod{\cgb 0 \cor 1 \cge 
     \cgb 0 \cor 1 \cge 
     \cgb 0 \cor 1 \cge
     0 0 0} \cnl \chs Flag on/off condition \\
\cod{$real_1$ $real_2$ $real_3$ 0.0 0.0 0.0} \cnl \chs Value (without curve?)\\
\cod{\cgb Live \cor Dead \cge} \cnl \chs Load type\\
\cod{Mid} \chs Reference surface
\end{quote}
\begin{enumerate}
\item Constant domain load referred to material frame works.
\item  There is no difference between \cod{Live}, \cod{Dead}. All load types
  refer to the undeformed/material 
  frame. 
\item Individual load curve is ignored : does not work.
\end{enumerate}


\subitempart{Dirichlet BC}\\
On nodes
\begin{quote}
%%\cod{------------------------------------DESIGN POINT DIRICH CONDITIONS}
\cod{E $int$} \cnl \chs Index of design entity\\
\cod{-} \cnl \\
\cod{\cgb 0 \cor 1 \cge \cgb 0 \cor 1 \cge \cgb 0 \cor 1 \cge 0 0
  0} \cnl \chs Flag on/off condition \\
\cod{$real_1$ $real_2$ $real_3$ 0.0 0.0 0.0} \cnl \chs Value\\
\cod{\cgb none \cor $int_1$ \cge 
\cgb none \cor $int_2$ \cge
\cgb none \cor $int_3$ \cge
none none none} \cnl
\chs Prescr. displ. by `Load curve' index\\
\cod{$int_1$ $int_2$ $int_3$ 0 0 0} \chs Spatial distribution `Function'
\end{quote}
\begin{enumerate}
\item Constant prescribed displacements work.
\item Load-Curve-prescribed displacements work, in this case the values
  $real_1$ etc are \emph{multiplied} by the load curve
\item Spatial `Function'-distribution???
\end{enumerate}

On lines
\begin{quote}
%%\cod{-------------------------------------DESIGN LINE DIRICH CONDITIONS}\\
\cod{E $int$} \cnl \chs Index of design entity\\
\cod{-} \cnl \\
\cod{\cgb 0 \cor 1 \cge 
     \cgb 0 \cor 1 \cge 
     \cgb 0 \cor 1 \cge 
     0 0 0} \cnl \chs Flag on/off condition \\
\cod{$real_1$ $real_2$ $real_3$ 0.0 0.0 0.0} \cnl \chs Value\\
\cod{\cgb none \cor $int_1$ \cge
     \cgb none \cor $int_2$ \cge
     \cgb none \cor $int_3$ 
     none none none} \cnl
\chs prescribed displacement `Load curve'\\
\cod{$int_1$ $int_2$ $int_3$ 0 0 0} \chs Spatial distribution `Function'
\end{quote}
\begin{enumerate}
\item Constant prescribed displacements work.
\item Load-Curve-prescribed displacements work.
\item Spatial `Function'-scaling does not work 
\end{enumerate}

On surface
\begin{quote}
%%\cod{-------------------------------------DESIGN SURF DIRICH CONDITIONS}\\
\cod{E $int$} \cnl \chs Index of design entity\\
\cod{-} \cnl \\
\cod{\cgb 0 \cor 1 \cge 
     \cgb 0 \cor 1 \cge
     \cgb 0 \cor 1 \cge
     0 0 0} \cnl \chs Flag on/off condition \\
\cod{$real_1$ $real_2$ $real_3$ 0.0 0.0 0.0} \cnl \chs Value\\
\cod{\cgb none \cor $int_1$\cge
     \cgb none \cor $int_2$ \cge
     \cgb none \cor $int_3$ \cge
     none none none} \cnl \chs prescribed displacement `Load curve'\\
\cod{$int_1$ $int_2$ $int_3$ 0 0 0} \chs Spatial distribution `Function'
\end{quote}
\begin{enumerate}
\item Constant prescribed displacements work.
\item Load-Curve-prescribed displacements work.
\item Spatial `Function'-distribution???
\end{enumerate}

On volume
\begin{quote}
%%\cod{--------------------------------------DESIGN VOL DIRICH CONDITIONS}\\
\cod{E $int$} \cnl \chs Index of design entity\\
\cod{-} \cnl \\
\cod{\cgb 0 \cor 1 \cge 
     \cgb 0 \cor 1 \cge
     \cgb 0 \cor 1 \cge
     0 0 0} \cnl \chs Flag on/off condition \\
\cod{$real_1$ $real_2$ $real_3$ 0.0 0.0 0.0} \cnl \chs Value\\
\cod{\cgb none \cor $int_1$\cge
     \cgb none \cor $int_2$ \cge
     \cgb none \cor $int_3$ \cge
     none none none} \cnl \chs prescribed displacement `Load curve'\\
\cod{$int_1$ $int_2$ $int_3$ 0 0 0} \chs Spatial distribution `Function'
\end{quote}
\begin{enumerate}
\item Constant prescribed displacements work.
\item Load-Curve-prescribed displacements work.
\item Spatial `Function'-distribution???
\end{enumerate}


\subitempart{Dirichlet and Neumann BC --- status summary}\\
Explanation: `+' means working, `-' is for \fail{not
  implemented}, `0' is undetermined, ` ' does not apply
\begin{center}
\begin{tabular}{c|cccc|cccc|c}
  & \multicolumn{4}{c|}{Dirichlet BC} & \multicolumn{4}{c|}{Neumann BC}
  & Solution
\\
  & nod. & lin. & surf. & vol.
  & nod. & lin. & surf. & vol.
  & tested with
\\ \hline
  Curves 
  & + & + & + & + 
  & - & - & - & - 
  & Dyn. gen.-$\alpha$
\\
  Functions 
  & 0 & - & 0 & 0
  &  &  & &
  & Dyn. gen.-$\alpha$
\end{tabular}
\end{center}

\subitempart{Gid input} Data : Conditions : Single Layer : Volumes : Brick

\subitempart{Gid output}\\
Displacements can be shown for any discretisation type

Stresses can only be displayed for these combinations
\begin{enumerate}
\item hex8 \& $2\times2\times2$ Gauss points
\item hex20 \& $3\times3\times3$ Gauss points
\end{enumerate}

\subitempart{src/Input files}
\cod{c1\_elast.dat},
\cod{c1\_elast\_geonln.dat},
\cod{c1\_elast\_geonln\_sd.dat},
\cod{c1\_mat\_hyperpolyconvex.dat},
\cod{c1\_neum\_geonln.dat},
\cod{c1\_optfsd.dat},
\cod{c1\_plast\_mises.dat}

\subitempart{Suggestions}
\begin{enumerate}
\item Tetrahedra $\longrightarrow$ SOLID3
\item Nodal interpolation of stresses is very inefficient, consider recoding
\end{enumerate}

%%
%%::::::::::::::::::::::::::::EBNF
\section{Extended Backus--Naur formalism}\label{brick1:sec:ebnf}
An extended Backus--Naur Formalism (EBNF), Reiser and Wirth
%%\cite{reiser94}
, is used to describe the input lines.

\begin{itemize}
\item Several listed construct are regarded as concatenated:\\
  \cod{C $=$ A B} means \cod{C} consists of \cod{A} followed by \cod{B}.
\item Alternatives are separated by $|$:\\
  \cod{C $=$ A $|$ B} means \cod{C $=$ A} or \cod{C $=$ B}, but not 
  \cod{C $=$ A B} or \cod{C $=$ $\emptyset$}\@. 
\item Brackets $[$ and $]$ denote optionality of the enclosed 
  construct:\\
  \cod{B $=$ $[$A$]$} results in \cod{B $=$ A} or \cod{B $=$ $\emptyset$}\@.
\item Braces $\{$ and $\}$ denote a repetition of a construct, 
  which includes zero repetitions:\\
  \cod{B $=$ $\{$A$\}$} is equivalent to $\cod{B} = \emptyset, \cod{A}, \cod{AA}, \cod{AAA}, \ldots$
\item Parentheses $($ and $)$ group expressions.
\item An ellipsis $\ldots$ represents reasonable continuation.
\end{itemize}