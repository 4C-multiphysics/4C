
\global\long\def\vol{\operatorname{vol}}
\global\long\def\dual#1#2{\langle#1,#2\rangle}
\global\long\def\field#1{\mathbb{#1}}
\global\long\def\R{\field R}
\global\long\def\set#1{\{#1\}}
\global\long\def\st{;\,}
\global\long\def\bigset#1{\bigl\{#1\bigr\}}
\global\long\def\Parens#1{\bigl(#1\bigr)}
\global\long\def\sprod#1#2{\langle#1,#2\rangle}
\global\long\def\dotid{\,\cdot\,}
\global\long\def\upd{\mathrm{d}}
\global\long\def\d{\,\upd}
\global\long\def\abs#1{\lvert#1\rvert}

\chapter{Surface areas and surface integrals}

This chapter tries to explain the background of the
\texttt{ComputeMetricTensorForSurface} function that can be found at
various places in \baci{}.

\paragraph{Motivation and definition.}

(a) Let $a^{1},\ldots,a^{k}\in\R^{n}$. Let\[
A:=\Parens{a^{1}\; a^{2}\;\cdots\; a^{k}}\in\R^{n\times k}\]
be the matrix with columns $a^{1},\ldots,a^{k}$. Then\[
A\Parens{[0,1]^{k}}=\bigset{Ax\st x\in\R^{k},\; x_{j}\in[0,1]\text{ für alle }j=1,\ldots,k}\subseteq\R^{n}\]
is the \textit{parallelepiped} spanned by the vectors
$a^{1},\ldots,a^{k}$. If $k=2$, this is the
\textit{parallelogram} spanned by the vectors $a^1$ and $a^2$.

(b) In the case $k=n$, it holds that $\vol_{n}\Parens{A\Parens{[0,1]^{n}}}=\abs{\det A}$.
If $a^{1},\ldots,a^{n}$ are linearly dependent, then boths sides
are $=0$, for $A\Parens{[0,1]^{k}}$ has width $0$ in (at least)
one direction.

(c) The \textit{Gramian matrix} of the vectors $a^{1},\ldots,a^{k}$ is defined as
 \[
  A^{\top}A = \Parens{\sprod{A^{\top}Ae_{i}}{e_{j}}}_{i,j=1,\ldots,n}
            = \Parens{\sprod{Ae_{i}}{Ae_{j}}}_{i,j=1,\ldots,n}
	    = \Parens{\sprod{a_{i}}{a_{j}}}_{i,j=1,\ldots,n}
            \in\R^{n\times n}
\]
(with $\sprod{\dotid}{\dotid}$ being the standard scalar product in
$\R^{k}$ and $e_{i}$ the $i$th standard basis vector of $\R^{k}$).
It is positive-semidefinite (because $\sprod{A^{\top}Ax}x=\abs{Ax}^{2}\ge0$
for all $x\in\R^{k}$) and therefore $\det A^{\top}A\ge0$.

Now we define\[
\gamma(A):=\sqrt{\det(A^{\top}A)}=\sqrt{\det\Parens{\sprod{Ae_{i}}{Ae_{j}}}}.\]
Notice that $\gamma(A)=\abs{\det A}$ if $k = n$.

(d) We want to motivate why $\gamma(A)$ is the $k$-dimensional volume
of $A\Parens{[0,1]^{k}}$. Here is an example: Let $n=k+1$ and
$A\Parens{[0,1]^{k}}\subseteq\R^{k}\times\set 0$, e.g. the parallelepiped
has width $0$ in the direction of $e_{n}=e_{k+1}$. Let $Q:\R^{k+1}\to\R^{k}$
be the projektion onto the first $k$ coordinates (thus $Q(x)=Q\Parens{(x_{1},\ldots,x_{n})}=(x_{1},\ldots,x_{k})$),
then\[
(QA)^{\top}QA=A^{\top}Q^{\top}QA=A^{\top}A,\]
and therefore $\vol_{k}(QA\Parens{[0,1]^{k}})=\abs{\det(QA)}=\gamma(A)$,
using part (b). So in this case, $\gamma(A)$ is indeed the $k$-dimensional
volume of $A\Parens{[0,1]^{k}}$.

\medskip{}



\paragraph{Integration on submanifolds.}

Now let $M\subseteq\R^{n}$ be a $k$-dimensional submanifold of $\R^{n}$,
with global parameterization. While we won't give the exact definition
of submanifolds here, this basically means that there is some open
set $\Omega\subseteq\R^{k}$ and a parameterization $\Phi:\Omega\to M$
that is, among other things, smooth and one-to-one.
Also, let $f:M\to\R$ be a suitable function (again, we don't give the
exact requirements here). Then we define the \textit{surface integral}\[
\int_{M}f(x)\d S(x) := \int_{\Omega}f(\Phi(\xi))\,\gamma(\Phi'(\xi))\d\xi.\]
In particular, we define\[
\vol_{k}(M):=\int_{M}1\d S(x) =\int_{\Omega}\gamma(\Phi'(\xi))\d\xi.\]


\paragraph{Example.} A parameterization%
\footnote{Actually, this is not a parameterization of all of the unit sphere:
The half-plane $\set{x\in\R^{3}\st x_{2}=0,x_{1}\le0}$ is missing.
A global parameterization of the unit sphere doesn't exist, and the
missing half-plane is a set of $2$-dimensonal measure zero, so we
can disregard it.%
} of the two-dimensional unit sphere $S_{2}:=\set{x\in\R^{3}\st\abs x=1}$
in $\R^{3}$ is\begin{gather*}
\Phi:(-\pi,\pi)\times(-\tfrac{\pi}{2},\tfrac{\pi}{2})\to\R^{3},\\
\Phi(\varphi_{1},\varphi_{2}):=\begin{pmatrix}\cos\varphi_{1}\cos\varphi_{2}\\
\sin\varphi_{1}\cos\varphi_{2}\\
\sin\varphi_{2}\end{pmatrix}.\end{gather*}
Its derivative (the Jacobian matrix) is\[
\Phi'(\varphi_{1},\varphi_{2})=\begin{pmatrix}-\sin\varphi_{1}\cos\varphi_{2} & -\cos\varphi_{1}\sin\varphi_{2}\\
\cos\varphi_{1}\cos\varphi_{2} & -\sin\varphi_{1}\sin\varphi_{2}\\
0 & \cos\varphi_{2}\end{pmatrix},\]
and thus\[
\Phi'(\varphi_{1},\varphi_{2})^{\top}\Phi'(\varphi_{1},\varphi_{2})=\begin{pmatrix}\cos^{2}\varphi_{2} & 0\\
0 & 1\end{pmatrix},\]
so we see that $\gamma(\Phi'(\varphi_{1},\varphi_{2}))=\cos\varphi_{2}$.
Now we compute the 2-dimensional volume, i.e. the surface area of
the unit sphere, by\[
\vol_{2}(S_{2})=\int_{S_{2}}1\d S(x)=\int_{\varphi_{1}=-\pi}^{\pi}\int_{\varphi_{2}=-\pi/2}^{\pi/2}\cos\varphi_{2}\d\varphi_{2}\d\varphi_{1}=\left.2\pi\sin\varphi_{2}\right|_{-\pi/2}^{\pi/2}=4\pi.\]

\paragraph{Remark.}

For vectors $a,b\in\R^{3}$ in the three-dimensonal space $\R^{3}$
it holds that\[
\abs{a\times b}^{2}=\sprod{a\times b}{a\times b}=\sprod aa\sprod bb-\sprod ab^{2}=\det\begin{pmatrix}\sprod aa & \sprod ab\\
\sprod ba & \sprod bb\end{pmatrix},\]
and thus if $\Phi(t)=\Phi(t_{1},t_{2})$, then\[
\abs{\partial_{t_{1}}\Phi\times\partial_{t_{2}}\Phi}^{2}=\det\Parens{\sprod{\partial_{t_{i}}\Phi}{\partial_{t_{j}}\Phi}_{i,j}}=\det(\Phi'^{\top}\Phi').\]
The matrix $\Phi'^{\top}\Phi'$, i.e. the Gramian matrix of the vectors
$\partial_{t_{1}}\Phi$ and $\partial_{t_{2}}\Phi$, is also called the \textit{metric
tensor}, and the expression \[
\abs{\partial_{t_{1}}\Phi\times\partial_{t_{2}}\Phi}\d t_{1}\d t_{2}=\gamma(\Phi')\d S\]
 is known in engineering as the \textit{area element.}
